\begin{preface}{May 2022}{Takayuki Ishizaki}

In Japan, the Great East Japan Earthquake of 2011 triggered the start of electric power system reform.
In 2012, the following year, the Ministry of Education, Culture, Sports, Science and Technology (MEXT) launched the "Creation and Integration of Theories and Fundamental Technologies for the Construction of Distributed and Cooperative Energy Management Systems" as a project-type research project by the Japan Science and Technology Agency (JST).
The authors, Ishizaki and Kawaguchi, are researchers who participated in the project, specializing in systems and control engineering, and were engaged in the research and development of renewable energy as a main power source.

The impetus for writing this book came from our own experience of the "high barriers to entry for power system R\&D" in the above project.
At the beginning of the project, we were complete amateurs in the field of power systems.
We started learning the basics of related fields to conduct our research, but we could not understand most of the things that we should have been familiar with in systems and control engineering, such as differential equations and optimization, in the context of power system engineering.
The reason for this was that power system engineering is a practical academic discipline that emphasizes the reality of power systems, while systems control engineering is a mathematical discipline with its origins in mathematics, and thus has very different academic values and styles.
This difference in awareness of the underlying issues is also thought to be a factor that hinders practical communication between researchers in systems and control engineering and those in power system engineering.
As our daily research activities revealed this hindsight, we came to strongly feel the necessity of bridging the gap between the disciplines.

This book is intended for students, researchers, and engineers in the field of systems and control who wish to work on research and development of power systems with the purpose of:
\begin{itemize}
\item To be able to understand the structural and mathematical basis of power systems in the language of systems and control engineering
\item To be able to independently build a numerical simulation environment for power system analysis and control
\end{itemize}
Specifically, for the former purpose, the configuration and characteristics of power systems are explained using basic concepts such as equations of state, equilibrium points, stability, and feedback in systems and control engineering.
The prerequisite knowledge required for this course is linear algebra, dynamical systems theory, and AC circuit theory from liberal arts courses at universities.
For the latter purpose, in addition to illustrating scripts for running numerical simulations, the fundamentals of object-oriented thinking and programming techniques that describe the entire complex power system as a group of modules divided by function will be explained.
A power system is not a single mechanical system, but a synthesis of a wide variety of elements and many decision makers.
Therefore, the numerical simulation environment should be modularized accordingly.
A distributed development environment structured into a group of modules is also very useful as a common base for aggregating the knowledge of multiple people.
Although this document assumes implementation in \matlab, the core ideas should be applicable to other programming languages as well.

Finally, it should be noted that all authors have made every effort to ensure that the descriptions throughout this document are as accurate as possible.
In particular, the author, Kawabe, checked the terminology and wording related to power systems from a professional perspective in power system engineering.
In addition, students and collaborators in the authors' laboratories pointed out errors and unclear descriptions, and the authors have worked hard to improve the manuscript.
The authors would like to thank Masahiro Ito, Taku Nishino, Miki Taya, Yota Kimura, Yoshihito Kinoshita, Yoshiyuki Onishi, Taichi Ichimura, Kentaro Oda, and Hirohiro Kawano for their significant contributions in writing and improving the manuscript.

We hope that this book will be a catalyst for new learning for those involved in systems and control engineering and help support the future of electric power systems.

\end{preface}