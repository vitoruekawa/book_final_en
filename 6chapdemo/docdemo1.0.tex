\documentclass[tombow,dvipdfmx]{corona-a5-1.1}
% dvipdfmxを追加(川口)

% Springer document settings
\usepackage[bottom]{footmisc}% places footnotes at page bottom

\usepackage{newtxtext}       % 
\usepackage[varvw]{newtxmath}       % selects Times Roman as basic font
%%%%%%%%%%%%%%%%%%%%%%%%%%%%%%%

% \usepackage{amssymb}
\usepackage{amsmath}
\usepackage{enumitem}


\usepackage{graphicx}
\usepackage{color}
\usepackage{cite}
\usepackage{makeidx}
\usepackage{ntheorem}

\usepackage{ascmac}
\usepackage{eclbkbox}
\usepackage{dsfont}

\usepackage{longtable}

\usepackage{url}

\usepackage{hyperref}

\usepackage{multicol}

%% --川口追加--
\makeatletter
\let\MYcaption\@makecaption
\makeatother
\usepackage{subcaption}
\captionsetup{compatibility=false}      % 必要に応じて

\makeatletter
\let\@makecaption\MYcaption
\makeatother
% ----

%%
\theoremstyle{plain}
\theoremheaderfont{\bfseries}
\theorembodyfont{\rmfamily}
\theoremseparator{\hspace{1ex}}
\theoremindent0cm
\theoremnumbering{arabic}
\theoremprework{\vspace{1ex}\begin{shadebox}\vspace{1ex}}
\theorempostwork{\vspace{-1ex}\end{shadebox}\vspace{1ex}}

%%
\theoremclass{theorem}

%%
\theoremclass{theorem}

%%
\theoremclass{theorem}


%%
\theoremstyle{break}
\theoremheaderfont{\bfseries}
\theorembodyfont{\rmfamily}
\theoremseparator{}
\theoremindent0cm
\theoremnumbering{arabic}
\theoremprework{\vspace{1.5ex}\begin{breakbox}\vspace{-0.5ex}}
\theorempostwork{\vspace{-0.5ex}\end{breakbox}\vspace{1.5ex}}

%%
\theoremstyle{nonumberplain}
\theoremseparator{\hspace{1ex}}

%%
\newtheorem{assumption}{Assumption}[section]

%%
\renewcommand{\theproblem}{}

\renewcommand{\theremark}{}


\newcommand{\red}[1]{{\color{red}#1}}
\newcommand{\blue}[1]{{\color{blue}#1}}
\newcommand{\green}[1]{{\color{green}#1}}

\DeclareMathOperator*{\argmax}{arg\,max}

\newcommand{\bm}[1]{\boldsymbol{#1}}
\newcommand{\sfT}{\mathsf{T}}

\newcommand{\advanced}{$^{\ddag}$}

\DeclareMathOperator{\sfsin}{\mathsf{sin}}
\DeclareMathOperator{\sfcos}{\mathsf{cos}}
\DeclareMathOperator{\sftan}{\mathsf{tan}}
\DeclareMathOperator{\sfarctan}{\mathsf{arctan}}

\DeclareMathOperator{\sfdiag}{\mathsf{diag}}
\DeclareMathOperator{\sfcol}{\mathsf{col}}
\DeclareMathOperator{\sfdet}{\mathsf{det}}
\DeclareMathOperator{\sfadj}{\mathsf{adj}}
\DeclareMathOperator{\sftrace}{\mathsf{trace}}

\DeclareMathOperator{\real}{\mathsf{Re}}

\DeclareMathOperator{\sfker}{\mathsf{ker}}
\DeclareMathOperator{\sfim}{\mathsf{im}}

\DeclareMathOperator{\sfdim}{\mathsf{dim}}
\DeclareMathOperator{\sfspan}{\mathsf{span}}

\DeclareMathOperator{\sfint}{\mathsf{int}}

\DeclareMathOperator*{\sfmin}{\mathsf{min}}
\DeclareMathOperator*{\sfmax}{\mathsf{max}}
\DeclareMathOperator*{\sfsup}{\mathsf{sup}}

\DeclareMathOperator{\sfsat}{\mathsf{sat}}

\newcommand{\mat}[1]{\left[\: \begin{matrix} #1 \end{matrix} \:\right]}
\newcommand{\spliteq}[1]{\begin{split} #1 \end{split}}
\newcommand{\simode}[1]{\begin{cases}  \begin{split} #1 \end{split} \end{cases}}

\newcommand{\proofend}{\hfill \rule{2mm}{3mm}}

\newcommand{\Xti}{X_i'}
\newcommand{\Xsi}{X_i}

\newcommand{\Xtone}{X_1'}
\newcommand{\XtN}{X_N'}

\newcommand{\Xt}{X'}
\newcommand{\Xs}{X}

\newcommand{\taudi}{\tau_i}
\newcommand{\taud}{\tau}

\newcommand{\Cgi}{b_i}


\newcommand{\Ifd}{I_{\rm field} }

\newcommand{\matlab}{\textsc{Matlab} }





%% --川口追加--
\newcommand{\thshift}{\theta_{12}}
\newcommand{\thshiftb}{\theta_{32}}
\newcommand{\Ysa}{\bm y_{12}}
\newcommand{\bca}{c_{12}}
\newcommand{\Ysb}{\bm y_{32}}
\newcommand{\bcb}{c_{32}}
\newcommand{\bcij}{c_{ij}}
\newcommand{\Is}{{\bm I}_{12}' }
\newcommand{\im}{\bm j}
\newcommand{\tr}{{\sf T}}

\begin{document}

\chapter{大規模モデルの数値シミュレーション例}\label{chap:largesim}

本節では,
◯◯◯◯◯◯◯◯◯◯◯◯◯◯◯◯◯◯◯◯◯◯◯◯◯◯◯◯◯◯◯◯◯◯◯
◯◯◯◯◯◯◯◯◯◯◯◯◯◯◯◯◯◯◯◯◯◯◯◯◯◯◯◯◯◯◯◯◯◯◯
◯◯◯◯◯◯◯◯◯◯◯◯◯◯◯◯◯◯◯◯◯◯◯◯◯◯◯◯◯◯◯◯◯◯◯
◯◯◯◯◯◯◯◯◯◯◯◯◯◯◯◯◯◯◯◯◯◯◯◯◯◯◯◯◯◯◯◯◯◯◯
◯◯◯◯◯◯◯◯◯◯◯◯◯◯◯◯◯◯◯◯◯◯◯◯◯◯◯◯◯◯◯◯◯◯◯
◯◯◯◯◯◯◯◯◯◯◯◯◯◯◯◯◯◯◯◯◯◯◯◯◯◯◯◯◯◯◯◯◯◯◯
◯◯◯◯◯◯◯◯◯◯◯◯◯◯◯◯◯◯◯◯◯◯◯◯◯◯◯◯◯◯◯◯◯◯◯
◯◯◯◯◯◯◯◯◯◯◯◯◯◯◯◯◯◯◯◯◯◯◯◯◯◯◯◯◯◯◯◯◯◯◯


\section{対象とする電力系統モデル}

\begin{figure}[t]
\centering
\includegraphics[width = .99\linewidth]{figs/IEEE68bus}
\medskip
\caption{\textbf{IEEE68母線系統モデル}}
\label{fig:IEEE68bus}
\medskip
\end{figure}

\subsection{IEEE68母線系統モデル}

本節では,IEEE68母線系統モデル(\FIGref{fig:IEEE68bus})を用いた数値シミュレーション結果を例示する。
このモデルは,68地点の母線で構成されており,そのうち16地点の母線に発電機,35地点の母線に負荷が接続されている。
\FIGref{fig:IEEE68bus}における「エリア1」は米国北東部のニューイングランド地域の電力系統を表しており,「エリア2」はニューヨーク州の電力系統を表している。
また,エリア3からエリア5は,ニューヨーク州周辺の電力系統をそれぞれ1組の発電機と負荷に集約して表現している。



送電線は,\ref{sec:transmodc}節で説明した対地静電容量を考慮したモデルとして設定する。
各送電線の定数は\ref{table:ieee68lines}の値に設定する。
1列目は送電線端の母線番号,2列目,3列目,4列目はそれぞれ,送電線の抵抗,リアクタンス,対地静電容量の値を表す。
これらは\cite[Appendix A]{pal2006robust}に示されている標準的な値である。
また,発電機には\ref{sec:genmodadv}節で説明した突極型の発電機モデルを採用する。
各発電機の定数は\ref{table:ieee68genpara}の値に設定する。
これらも\cite[Appendix B]{pal2006robust}に示されている値である。



\subsection{潮流計算に用いるデータシート}


潮流計算に用いる発電機母線のデータシートを\ref{table:ieee68datag}に示す。
同様に,負荷母線のデータシートを\ref{table:ieee68datal}に示す。
これらも\cite[Appendix A]{pal2006robust}に示されている値を引用している。


\subsection{負荷モデル}

負荷モデルには\ref{sec:loadpr}節で説明した定インピーダンスモデルを採用する。
\ref{table:ieee68datag}と\ref{table:ieee68datal}のデータシートに対して,潮流計算の結果から定められる負荷のインピーダンス値は\ref{table:ieee68loads}となる。
なお,母線xxをスラック母線に設定した。





\section{負荷変動に対する周波数安定性解析}

\subsection{負荷変動の設定}

本節では,負荷のインピーダンス値を変化させた場合の周波数偏差の時間応答を観察する。
負荷のインピーダンスは,\ref{table:ieee68loads}の値を基準として,1時間あたり10\%ずつ線形に増加させる。
すなわち,各負荷のインピーダンスは
\begin{align}\label{eq:loadvary68}
z_{{\rm load}i}(t) = \left(1+\tfrac{1}{36000}t \right) \left( r_{{\rm load}i} + \bm{j} x_{{\rm load}i} \right)
\end{align}
と設定する。
なお,時刻$t$の単位は[s]である。

\subsection{発電機の機械的トルクと界磁電圧が定数である場合}\label{sec:constPV}

発電機の外部入力である機械的トルクと界磁電圧がすべて定数である場合を考える。
\ref{table:ieee68datag}と\ref{table:ieee68datal}のデータシートに対する定常潮流状態から,式\ref{eq:loadvary68}にしたがって負荷のインピーダンスが変化した場合の周波数偏差の時間応答をxxx(a)に示す。
この場合には,負荷変動が生じると需給バランスが満たされないため,周波数偏差は0にならないことがわかる。
また,負荷のインピーダンス値の増加に伴う消費電力の増加のため,発電機の制御を行わない場合には,周波数偏差が負の方向に増大していくこともわかる。

\subsection{自動電圧調整器のみを組み込む場合}\label{sec:onlyAVR}

発電機の界磁電圧を制御する自動電圧調整器が,各々すべての発電機に組み込まれた場合を考える。
自動電圧調整器は,式\ref{eq:avrst1}のIEEE ST1型モデルに設定する。
また,それらのパラメータはすべて共通であり,\ref{table:AVRparast1}の1行目の値に設定するものとする。
この場合の周波数偏差の時間応答をxxx(b)に示す。
xxx(b)と同様に,負荷変動のため需給バランスが満たされず,周波数偏差は0にならないことがわかる。
なお,発電機の機械的トルクは,\ref{sec:constPV}節と同様にすべて定数としている。


\subsection{自動発電制御器を同時に組み込む場合}

発電機の機械的トルクを制御する自動発電制御器を組み込む場合を考える。
ここでは,\FIGref{fig:IEEE68bus}の各々すべてのエリアに対して,\ref{sec:broadPI}節で説明したブロードキャスト型PIコントローラを組み込む。
具体的には,エリア$l$に属する発電機母線の添字集合を$\mathcal{I}^{(l)}_{\rm G}$と表すと,エリア$l$の自動発電制御器は
\begin{align}
\simode{
\dot{\xi}^{(l)}&=  \Delta \omega_{\rm sum}^{(l)} \\
P_{{\rm mech}i} &= P_{{\rm mech}i}^{{\star}} 
- \alpha_i \left(k_{\rm P}^{(l)} \Delta \omega_{\rm sum}^{(l)} +  k_{\rm I}^{(l)} \xi^{(l)} \right)
}
\qquad
i \in \mathcal{I}^{(l)}_{\rm G}
\end{align}
で与えられる。
ただし,エリア$l$に属する発電機の周波数偏差の重み付き和を
\[
\Delta \omega_{\rm sum}^{(l)}(t) := 
\sum_{i \in \mathcal{I}^{(l)}_{\rm G}} \alpha_i \Delta \omega_{i}(t)
\]
と定義している。
また,$k_{\rm P}^{(l)}$,$k_{\rm I}^{(l)}$はコントローラゲインである。
なお,自動電圧調整器は,\ref{sec:onlyAVR}節と同じものが各発電機に組み込まれているものとする。

この場合の周波数偏差の時間応答をxxx(c)に示す。
自動発電制御の働きによって,周波数偏差は0付近の値に維持されていることがわかる。
なお,負荷のインピーダンス値が継続して変化しているため周波数偏差は厳密に0にはならず,xxx~[pu]程度の小さな値となっている。

つぎに,初期時刻から3時間が経過するまでの発電機変数の変化を\FIGref{}に示す。
\FIGref{}(a)--(d)はそれぞれ,\red{$E e^{\bm{j} \delta_i}$},$\bm{V}_i$,$\bm{I}_i$,$P_i+\bm{j}Q_i$の変化を複素平面上に図示したものである。
初期時刻$t=0$における値を丸の印,終端時刻$t=10800$における値を四角の印で表示している。
この図から以下が読み取れる。

\begin{itemize}
\item \FIGref{}(a):すべての発電機の内部電圧は\red{xxx~[pu]程度の大きさに維持されており},回転子偏角は時計回りに緩やかに変化している。
\item \FIGref{}(b):母線電圧の絶対値は1~[pu]付近で維持されており,位相は回転子偏角と同様に時計回りに緩やかに変化している。
\item \FIGref{}(c):負荷のインピーダンス値の増加に伴う消費電力の増加に合わせて,母線電流の絶対値が増加している。
\item \FIGref{}(d):負荷のインピーダンス値の増加に合わせて,母線に供給される有効電力と無効電力が増加している。
\end{itemize}

以上の結果から,自動発電制御によって周波数の安定化が適切に実現されていることがわかる。
また,緩やかな負荷変動に対しては,発電機の内部状態も振動することなく緩やかに変化すること,すなわち,電力系統全体が準定常状態にあることがわかる。

\section{母線地絡に対する過渡安定度解析}

\subsection{母線地絡の設定}

本節では,母線地絡に対する過渡安定度の解析を行う。
電力系統の過渡安定度は以下のように評価する。
母線$k$における地絡によって生じる発電機$i$の周波数偏差を$\Delta \omega_i^{(k)}$と表す。
このとき,母線$k$の地絡に対する電力系統全体の周波数偏差の感度を$\|\Delta \omega^{(k)}\|_{\mathcal{L}_2}$により評価する。
ただし,$\Delta \omega^{(k)}$はすべての発電機の周波数偏差$\Delta \omega_1^{(k)},\ldots,\Delta \omega_{16}^{(k)}$を並べたベクトルである。
また,すべての発電機母線と負荷母線に関して$\|\Delta \omega^{(k)}\|_{\mathcal{L}_2}$の値を並べた集合をそれぞれ
\begin{align}\label{eq:JGJL}
\mathcal{J}_{\rm G}:=
\left\{
\| \Delta \omega^{(k)} \|_{\mathcal{L}_2}
\right\}_{k\in \{1,\ldots,16\} }
,\quad
\mathcal{J}_{\rm L}
:=
\left\{
\| \Delta \omega^{(k)} \|_{\mathcal{L}_2}
\right\}_{k\in \{17,\ldots,68\} }
\end{align}
%\begin{align}\label{eq:JGJL}
%\spliteq{
%\mathcal{J}_{\rm G}
%&:=
%\left\{
%\| \Delta \omega^{(1)} \|_{\mathcal{L}_2}
%,\ldots, 
%\| \Delta \omega^{(16)}\|_{\mathcal{L}_2}
%\right\}
%\\
%\mathcal{J}_{\rm L}
%&:=
%\left\{
%\| \Delta \omega^{(17)} \|_{\mathcal{L}_2}
%,\ldots, 
%\| \Delta \omega^{(68)}\|_{\mathcal{L}_2}
%\right\}
%}
%\end{align}
と定義する。
地絡は特定の母線のみではなく,すべての母線に生じる可能性があることを考慮すると,母線地絡に対する過渡安定度は,データ集合$\mathcal{J}_{\rm G}$と$\mathcal{J}_{\rm L}$が適切な意味で小さいこととして評価できる。
本節では,$\mathcal{J}_{\rm G}$と$\mathcal{J}_{\rm L}$を箱ひげ図で表示することにより,それらの最大値,最小値,四分位数を用いて過渡安定度の比較に用いる。
なお,母線地絡の継続時間はすべて70~[ms]に設定する。


\subsection{自動電圧調整器を組み込む場合}

各発電機に自動電圧調整器を組み込んだ場合の過渡安定度を解析する。
それぞれの自動電圧調整器の設定は,\ref{sec:onlyAVR}節と同様であるものとする。
比較のため
\begin{itemize}
\item[(A)] 自動電圧調整器を組み込まない場合
\item[(B)] エリア1に属するすべての発電機に自動電圧調整器を組み込む場合
\item[(C)] エリア1およびエリア2に属するすべての発電機に自動電圧調整器を組み込む場合
\item[(D)] すべての発電機に自動電圧調整器を組み込む場合
\end{itemize}
の4通りの場合における式\ref{eq:JGJL}のデータ集合$\mathcal{J}_{\rm G}$と$\mathcal{J}_{\rm L}$を計算する。

得られたデータ集合を箱ひげ図として表示した結果を\FIGref{}に示す。
\FIGref{}(a)が発電機母線の地絡に関する$\mathcal{J}_{\rm G}$であり,\FIGref{}(b)が負荷母線の地絡に関する$\mathcal{J}_{\rm L}$である。
横軸は上記の(A)--(D)の場合分けに対応しており,それぞれの場合において左側の青色の箱ひげ図が自動電圧調整器を組み込んだ場合の結果である。
この結果から...

\subsection{系統安定化装置を同時に組み込む場合}

\ref{sec:retrofit}節で説明したレトロフィット制御理論に基づく系統安定化装置を各発電機に組み込んだ場合の過渡安定度を解析する。



\begin{figure}[t]
  \centering
  {
  \begin{minipage}{0.49\linewidth}
    \centering
    \includegraphics[width = 1.0\linewidth]{figs/boxplotgen}
    \subcaption{ $\mathcal{J}_{\rm G}$}
  \end{minipage}
  \begin{minipage}{0.49\linewidth}
    \centering
    \includegraphics[width = 1.0\linewidth]{figs/boxplotload}
    \subcaption{ $\mathcal{J}_{\rm L}$ }
  \end{minipage}
  \medskip
  \caption{\textbf{消費電力の増加に対する周波数偏差の時間応答} }
  \label{fig:agcPdemo}
  }
\medskip
\end{figure}



\begin{table}[h]
\medskip
\caption{\textbf{送電線の物理定数}} \label{table:ieee68lines}
 \centering
  {
  \begin{minipage}{0.49\linewidth}
    \centering
  \begin{tabular}{crrrrcc}
   \hline
$i$--$j$ & $r_{ij}$  &  $x_{ij}$ & $c_{ij}$ \\
   \hline \hline
1--54  & 0 & 0.0181 & 0 \\
2--58  & 0 & 0.0250 & 0 \\
3--62  & 0 & 0.0200 & 0 \\
4--19  & 0.0007 & 0.0142 & 0 \\
5--20  & 0.0009 & 0.0180 & 0 \\
6--22  & 0 & 0.0143 & 0 \\
7--23  & 0.0005 & 0.0272 & 0 \\
8--25  & 0.0006 & 0.0232 & 0 \\
9--29  & 0.0008 & 0.0156 & 0 \\
10--31  & 0 & 0.0260 & 0 \\
11--32  & 0 & 0.0130 & 0 \\
12--36  & 0 & 0.0075 & 0 \\
13--17  & 0 & 0.0033 & 0 \\
14--41  & 0 & 0.0015 & 0 \\
15--42  & 0 & 0.0015 & 0 \\
16--18  & 0 & 0.0030 & 0 \\
17--36  & 0.0005 & 0.0045 & 0.3200 \\
17--43  & 0.0005 & 0.0276 & 0 \\
18--42  & 0.0040 & 0.0600 & 2.2500 \\
18--49  & 0.0076 & 0.1141 & 1.1600 \\
19--20  & 0.0007 & 0.0138 & 0 \\
19--68  & 0.0016 & 0.0195 & 0.3040 \\
21--22  & 0.0008 & 0.0140 & 0.2565 \\
21--68  & 0.0008 & 0.0135 & 0.2548 \\
22--23  & 0.0006 & 0.0096 & 0.1846 \\
23--24  & 0.0022 & 0.0350 & 0.3610 \\
24--68  & 0.0003 & 0.0059 & 0.0680 \\
25--26  & 0.0032 & 0.0323 & 0.5310 \\
25--54  & 0.0070 & 0.0086 & 0.1460 \\
26--27  & 0.0014 & 0.0147 & 0.2396 \\
26--28  & 0.0043 & 0.0474 & 0.7802 \\
26--29  & 0.0057 & 0.0625 & 1.0290 \\
27--37  & 0.0013 & 0.0173 & 0.3216 \\
27--53  & 0.0320 & 0.3200 & 0.4100 \\
28--29  & 0.0014 & 0.0151 & 0.2490 \\
30--31  & 0.0013 & 0.0187 & 0.3330 \\
30--32  & 0.0024 & 0.0288 & 0.4880 \\
30--53  & 0.0008 & 0.0074 & 0.4800 \\
30--61  & 0.0019 & 0.0183 & 0.2900 \\
30--61  & 0.0019 & 0.0183 & 0.2900 \\
31--38  & 0.0011 & 0.0147 & 0.2470 \\
31--53  & 0.0016 & 0.0163 & 0.2500 \\
32--33  & 0.0008 & 0.0099 & 0.1680 \\
\hline
  \end{tabular}
%  \subcaption{データシート}
  \end{minipage}
  \begin{minipage}{0.49\linewidth}
    \centering
  \begin{tabular}{crrrcc}
   \hline
$i$--$j$ & $r_{ij}$  &  $x_{ij}$ & $c_{ij}$ \\
   \hline \hline
33--34  & 0.0011 & 0.0157 & 0.2020 \\
33--38  & 0.0036 & 0.0444 & 0.6930 \\
34--35  & 0.0001 & 0.0074 & 0 \\
34--36  & 0.0033 & 0.0111 & 1.4500 \\
35--45  & 0.0007 & 0.0175 & 1.3900 \\
36--61  & 0.0022 & 0.0196 & 0.3400 \\
36--61  & 0.0022 & 0.0196 & 0.3400 \\
37--52  & 0.0007 & 0.0082 & 0.1319 \\
37--68  & 0.0007 & 0.0089 & 0.1342 \\
38--46  & 0.0022 & 0.0284 & 0.4300 \\
39--44  & 0 & 0.0411 & 0 \\
39--45  & 0 & 0.0839 & 0 \\
40--41  & 0.0060 & 0.0840 & 3.1500 \\
40--48  & 0.0020 & 0.0220 & 1.2800 \\
41--42  & 0.0040 & 0.0600 & 2.2500 \\
43--44  & 0.0001 & 0.0011 & 0 \\
44--45  & 0.0025 & 0.0730 & 0 \\
45--51  & 0.0004 & 0.0105 & 0.7200 \\
46--49  & 0.0018 & 0.0274 & 0.2700 \\
47--48  & 0.0025 & 0.0268 & 0.4000 \\
47--48  & 0.0025 & 0.0268 & 0.4000 \\
47--53  & 0.0013 & 0.0188 & 1.3100 \\
50--51  & 0.0009 & 0.0221 & 1.6200 \\
52--55  & 0.0011 & 0.0133 & 0.2138 \\
53--54  & 0.0035 & 0.0411 & 0.6987 \\
54--55  & 0.0013 & 0.0151 & 0.2572 \\
55--56  & 0.0013 & 0.0213 & 0.2214 \\
56--57  & 0.0008 & 0.0128 & 0.1342 \\
56--66  & 0.0008 & 0.0129 & 0.1382 \\
57--58  & 0.0002 & 0.0026 & 0.0434 \\
58--59  & 0.0006 & 0.0092 & 0.1130 \\
57--60  & 0.0008 & 0.0112 & 0.1476 \\
59--60  & 0.0004 & 0.0046 & 0.0780 \\
60--61  & 0.0023 & 0.0363 & 0.3804 \\
58--63  & 0.0007 & 0.0082 & 0.1389 \\
62--63  & 0.0004 & 0.0043 & 0.0729 \\
62--65  & 0.0004 & 0.0043 & 0.0729 \\
63--64  & 0.0016 & 0.0435 & 0 \\
64--65  & 0.0016 & 0.0435 & 0 \\
65--66  & 0.0009 & 0.0101 & 0.1723 \\
66--67  & 0.0018 & 0.0217 & 0.3660 \\
67--68  & 0.0009 & 0.0094 & 0.1710 \\
\\
   \hline
  \end{tabular}
%   \subcaption{潮流計算結果}
  \end{minipage}
  }
\end{table}


\begin{table}[h]
\medskip
\caption{\textbf{発電機の物理定数}} \label{table:ieee68genpara}
 \centering
 {
\begin{tabular}{crrrrrr}
   \hline
\multicolumn{1}{c}{$i$} & \multicolumn{1}{r}{$M_{i}$} & \multicolumn{1}{r}{$D_{i}$} & \multicolumn{1}{r}{$\tau_{i}$} & \multicolumn{1}{r}{$X_{{\rm d}i}$} & \multicolumn{1}{r}{$X_{{\rm q}i}$} & \multicolumn{1}{r}{$X_{{\rm d}i}'$}  \\
   \hline \hline
1  & \multicolumn{1}{r}{42.0}  & \multicolumn{1}{r}{4.00} & \multicolumn{1}{r}{10.20} & \multicolumn{1}{r}{0.100} & \multicolumn{1}{r}{0.069} & \multicolumn{1}{r}{0.031} \\
2  & \multicolumn{1}{r}{30.2}  & \multicolumn{1}{r}{9.75} & \multicolumn{1}{r}{6.56} & \multicolumn{1}{r}{0.295} & \multicolumn{1}{r}{0.282} & \multicolumn{1}{r}{0.070} \\
3  & \multicolumn{1}{r}{35.8}  & \multicolumn{1}{r}{10.00} & \multicolumn{1}{r}{5.70} & \multicolumn{1}{r}{0.250} & \multicolumn{1}{r}{0.237} & \multicolumn{1}{r}{0.053} \\
4  & \multicolumn{1}{r}{28.6}  & \multicolumn{1}{r}{10.00} & \multicolumn{1}{r}{5.69} & \multicolumn{1}{r}{0.262} & \multicolumn{1}{r}{0.258} & \multicolumn{1}{r}{0.044} \\
5  & \multicolumn{1}{r}{26.0}  & \multicolumn{1}{r}{3.00} & \multicolumn{1}{r}{5.40} & \multicolumn{1}{r}{0.330} & \multicolumn{1}{r}{0.310} & \multicolumn{1}{r}{0.066} \\
6  & \multicolumn{1}{r}{34.8}  & \multicolumn{1}{r}{10.00} & \multicolumn{1}{r}{7.30} & \multicolumn{1}{r}{0.254} & \multicolumn{1}{r}{0.241} & \multicolumn{1}{r}{0.050} \\
7  & \multicolumn{1}{r}{26.4}  & \multicolumn{1}{r}{8.00} & \multicolumn{1}{r}{5.66} & \multicolumn{1}{r}{0.295} & \multicolumn{1}{r}{0.292} & \multicolumn{1}{r}{0.049} \\
8  & \multicolumn{1}{r}{24.3}  & \multicolumn{1}{r}{9.00} & \multicolumn{1}{r}{6.70} & \multicolumn{1}{r}{0.290} & \multicolumn{1}{r}{0.280} & \multicolumn{1}{r}{0.057} \\
9  & \multicolumn{1}{r}{34.5}  & \multicolumn{1}{r}{14.00} & \multicolumn{1}{r}{4.79} & \multicolumn{1}{r}{0.211} & \multicolumn{1}{r}{0.205} & \multicolumn{1}{r}{0.057} \\
10 & \multicolumn{1}{r}{31.0}  & \multicolumn{1}{r}{5.56} & \multicolumn{1}{r}{9.37} & \multicolumn{1}{r}{0.169} & \multicolumn{1}{r}{0.115} & \multicolumn{1}{r}{0.046} \\
11 & \multicolumn{1}{r}{28.2}  & \multicolumn{1}{r}{13.60} & \multicolumn{1}{r}{4.10} & \multicolumn{1}{r}{0.128} & \multicolumn{1}{r}{0.123} & \multicolumn{1}{r}{0.018} \\
12 & \multicolumn{1}{r}{92.3}  & \multicolumn{1}{r}{13.50} & \multicolumn{1}{r}{7.40} & \multicolumn{1}{r}{0.101} & \multicolumn{1}{r}{0.095} & \multicolumn{1}{r}{0.031} \\
13 & \multicolumn{1}{r}{248.0}  & \multicolumn{1}{r}{33.00} & \multicolumn{1}{r}{5.90} & \multicolumn{1}{r}{0.030} & \multicolumn{1}{r}{0.029} & \multicolumn{1}{r}{0.006} \\
14 & \multicolumn{1}{r}{300.0}  & \multicolumn{1}{r}{100.00} & \multicolumn{1}{r}{4.10} & \multicolumn{1}{r}{0.018} & \multicolumn{1}{r}{0.017} & \multicolumn{1}{r}{0.003} \\
15 & \multicolumn{1}{r}{300.0}  & \multicolumn{1}{r}{100.00} & \multicolumn{1}{r}{4.10} & \multicolumn{1}{r}{0.018} & \multicolumn{1}{r}{0.017} & \multicolumn{1}{r}{0.003} \\
16 & \multicolumn{1}{r}{225.0}  & \multicolumn{1}{r}{50.00} & \multicolumn{1}{r}{7.80} & \multicolumn{1}{r}{0.036} & \multicolumn{1}{r}{0.033} & \multicolumn{1}{r}{0.007} \\
\hline
\end{tabular}
}
\end{table}





\begin{table}[h]
\medskip
\caption{\textbf{潮流計算に用いるデータシート(発電機母線)}} \label{table:ieee68datag}
 \centering
  {
  \begin{minipage}{0.32\linewidth}
    \centering
  \begin{tabular}{crrcc}
   \hline
$i$ & $P_i^{\star}$  &  $|\bm{V}_i^{\star} |$ \\
   \hline \hline
1  & \multicolumn{1}{r}{2.50}   & \multicolumn{1}{r}{1.045} \\
2  & \multicolumn{1}{r}{5.45}   & \multicolumn{1}{r}{0.980} \\
3  & \multicolumn{1}{r}{6.50}   & \multicolumn{1}{r}{0.983} \\
4  & \multicolumn{1}{r}{6.32}   & \multicolumn{1}{r}{0.997} \\
5  & \multicolumn{1}{r}{5.05}   & \multicolumn{1}{r}{1.011} \\
6  & \multicolumn{1}{r}{7.00}   & \multicolumn{1}{r}{1.050} \\
   \hline
  \end{tabular}
%  \subcaption{データシート}
  \end{minipage}
  \begin{minipage}{0.32\linewidth}
    \centering
  \begin{tabular}{ccccc}
   \hline
$i$ & $P_i^{\star}$  &  $|\bm{V}_i^{\star} |$ \\
   \hline \hline
7  & \multicolumn{1}{r}{5.60}   & \multicolumn{1}{r}{1.063} \\
8  & \multicolumn{1}{r}{5.40}   & \multicolumn{1}{r}{1.030} \\
9  & \multicolumn{1}{r}{8.00}   & \multicolumn{1}{r}{1.025} \\
10 & \multicolumn{1}{r}{5.00}   & \multicolumn{1}{r}{1.010} \\
11 & \multicolumn{1}{r}{10.00}   & \multicolumn{1}{r}{1.000} \\
12 & \multicolumn{1}{r}{13.50}   & \multicolumn{1}{r}{1.016} \\
   \hline
  \end{tabular}
%   \subcaption{潮流計算結果}
  \end{minipage}
    \begin{minipage}{0.32\linewidth}
    \centering
  \begin{tabular}{ccccc}
   \hline
$i$ & $P_i^{\star}$  &  $|\bm{V}_i^{\star} |$ \\
   \hline \hline
13  & \multicolumn{1}{r}{35.91}   & \multicolumn{1}{r}{1.011} \\
14  & \multicolumn{1}{r}{17.85}   & \multicolumn{1}{r}{1.000} \\
15  & \multicolumn{1}{r}{10.00}   & \multicolumn{1}{r}{1.000} \\
16  & \multicolumn{1}{r}{40.00}   & \multicolumn{1}{r}{1.000} \\
\\
\\
   \hline
  \end{tabular}
%   \subcaption{潮流計算結果}
  \end{minipage}
  }
\end{table}
 
 
\begin{table}[h]
\medskip
\caption{\textbf{潮流計算に用いるデータシート(負荷母線)}} \label{table:ieee68datal}
 \centering
  {
  \begin{minipage}{0.32\linewidth}
    \centering
  \begin{tabular}{crrcc}
   \hline
$i$ & $P_i^{\star}$  &  $Q_i^{\star}$ \\
   \hline \hline
\multicolumn{1}{c}{17}  & \multicolumn{1}{r}{$-$60.00} & \multicolumn{1}{r}{$-$3.00} \\
\multicolumn{1}{c}{18}  & \multicolumn{1}{r}{$-$24.70} & \multicolumn{1}{r}{$-$1.23}  \\
\multicolumn{1}{c}{19}  & \multicolumn{1}{r}{0} & \multicolumn{1}{r}{0}  \\
\multicolumn{1}{c}{20}  & \multicolumn{1}{r}{$-$6.80} & \multicolumn{1}{r}{$-$1.03}  \\
\multicolumn{1}{c}{21}  & \multicolumn{1}{r}{$-$2.74} & \multicolumn{1}{r}{$-$1.15}  \\
\multicolumn{1}{c}{22}  & \multicolumn{1}{r}{0} & \multicolumn{1}{r}{0}  \\
\multicolumn{1}{c}{23}  & \multicolumn{1}{r}{$-$2.48} & \multicolumn{1}{r}{$-$0.85} \\
\multicolumn{1}{c}{24}  & \multicolumn{1}{r}{$-$3.09} & \multicolumn{1}{r}{0.92} \\
\multicolumn{1}{c}{25}  & \multicolumn{1}{r}{$-$2.24} & \multicolumn{1}{r}{$-$0.47} \\
\multicolumn{1}{c}{26}  & \multicolumn{1}{r}{$-$1.39} & \multicolumn{1}{r}{$-$0.17}  \\
\multicolumn{1}{c}{27}  & \multicolumn{1}{r}{$-$2.81} & \multicolumn{1}{r}{$-$0.76} \\
\multicolumn{1}{c}{28}  & \multicolumn{1}{r}{$-$2.06} & \multicolumn{1}{r}{$-$0.28}  \\
\multicolumn{1}{c}{29}  & \multicolumn{1}{r}{$-$2.84} & \multicolumn{1}{r}{$-$0.27}  \\
\multicolumn{1}{c}{30}  & \multicolumn{1}{r}{0} & \multicolumn{1}{r}{0} \\
\multicolumn{1}{c}{31}  & \multicolumn{1}{r}{0} & \multicolumn{1}{r}{0} \\
\multicolumn{1}{c}{32}  & \multicolumn{1}{r}{0} & \multicolumn{1}{r}{0} \\
\multicolumn{1}{c}{33}  & \multicolumn{1}{r}{$-$1.12} & \multicolumn{1}{r}{0} \\
\multicolumn{1}{c}{34}  & \multicolumn{1}{r}{0} & \multicolumn{1}{r}{0} \\
   \hline
  \end{tabular}
%  \subcaption{データシート}
  \end{minipage}
  \begin{minipage}{0.32\linewidth}
    \centering
  \begin{tabular}{crrcc}
   \hline
$i$ & $P_i^{\star}$  &  $Q_i^{\star}$ \\
   \hline \hline
\multicolumn{1}{c}{35}  & \multicolumn{1}{r}{0} & \multicolumn{1}{r}{0} \\
\multicolumn{1}{c}{36}  & \multicolumn{1}{r}{$-$1.02} & \multicolumn{1}{r}{0.19} \\
\multicolumn{1}{c}{37}  & \multicolumn{1}{r}{0} & \multicolumn{1}{r}{0} \\
\multicolumn{1}{c}{38}  & \multicolumn{1}{r}{0} & \multicolumn{1}{r}{0} \\
\multicolumn{1}{c}{39}  & \multicolumn{1}{r}{$-$2.67} & \multicolumn{1}{r}{$-$0.13} \\
\multicolumn{1}{c}{40}  & \multicolumn{1}{r}{$-$0.65} & \multicolumn{1}{r}{$-$0.24} \\
\multicolumn{1}{c}{41}  & \multicolumn{1}{r}{$-$10.00} & \multicolumn{1}{r}{$-$2.50} \\
\multicolumn{1}{c}{42}  & \multicolumn{1}{r}{$-$11.50} & \multicolumn{1}{r}{$-$2.50} \\
\multicolumn{1}{c}{43}  & \multicolumn{1}{r}{0} & \multicolumn{1}{r}{0} \\
\multicolumn{1}{c}{44}  & \multicolumn{1}{r}{$-$2.68} & \multicolumn{1}{r}{$-$0.05} \\
\multicolumn{1}{c}{45}  & \multicolumn{1}{r}{$-$2.08} & \multicolumn{1}{r}{$-$0.21} \\
\multicolumn{1}{c}{46}  & \multicolumn{1}{r}{$-$1.51} & \multicolumn{1}{r}{$-$0.29} \\
\multicolumn{1}{c}{47}  & \multicolumn{1}{r}{$-$2.03} & \multicolumn{1}{r}{$-$0.33} \\
\multicolumn{1}{c}{48}  & \multicolumn{1}{r}{$-$2.41} & \multicolumn{1}{r}{$-$0.02} \\
\multicolumn{1}{c}{49}  & \multicolumn{1}{r}{$-$1.64} & \multicolumn{1}{r}{$-$0.29} \\
\multicolumn{1}{c}{50}  & \multicolumn{1}{r}{$-$1.00} & \multicolumn{1}{r}{1.47} \\
\multicolumn{1}{c}{51}  & \multicolumn{1}{r}{$-$3.37} & \multicolumn{1}{r}{1.22} \\
\multicolumn{1}{c}{52}  & \multicolumn{1}{r}{$-$1.58} & \multicolumn{1}{r}{$-$0.30} \\
   \hline
  \end{tabular}
%   \subcaption{潮流計算結果}
  \end{minipage}
    \begin{minipage}{0.32\linewidth}
    \centering
  \begin{tabular}{crrcc}
   \hline
$i$ & $P_i^{\star}$  &  $Q_i^{\star}$ \\
   \hline \hline
\multicolumn{1}{c}{53}  & \multicolumn{1}{r}{$-$2.52} & \multicolumn{1}{r}{$-$1.19} \\
\multicolumn{1}{c}{54}  & \multicolumn{1}{r}{0} & \multicolumn{1}{r}{0} \\
\multicolumn{1}{c}{55}  & \multicolumn{1}{r}{$-$3.22} & \multicolumn{1}{r}{$-$0.02} \\
\multicolumn{1}{c}{56}  & \multicolumn{1}{r}{$-$2.00} & \multicolumn{1}{r}{$-$0.74} \\
\multicolumn{1}{c}{57}  & \multicolumn{1}{r}{0} & \multicolumn{1}{r}{0} \\
\multicolumn{1}{c}{58}  & \multicolumn{1}{r}{0} & \multicolumn{1}{r}{0} \\
\multicolumn{1}{c}{59}  & \multicolumn{1}{r}{$-$2.34} & \multicolumn{1}{r}{$-$0.84} \\
\multicolumn{1}{c}{60}  & \multicolumn{1}{r}{$-$2.09} & \multicolumn{1}{r}{$-$0.71} \\
\multicolumn{1}{c}{61}  & \multicolumn{1}{r}{$-$1.04} & \multicolumn{1}{r}{$-$1.25} \\
\multicolumn{1}{c}{62}  & \multicolumn{1}{r}{0} & \multicolumn{1}{r}{0} \\
\multicolumn{1}{c}{63}  & \multicolumn{1}{r}{0} & \multicolumn{1}{r}{0} \\
\multicolumn{1}{c}{64}  & \multicolumn{1}{r}{$-$0.09} & \multicolumn{1}{r}{$-$0.88} \\
\multicolumn{1}{c}{65}  & \multicolumn{1}{r}{0} & \multicolumn{1}{r}{0} \\
\multicolumn{1}{c}{66}  & \multicolumn{1}{r}{0} & \multicolumn{1}{r}{0} \\
\multicolumn{1}{c}{67}  & \multicolumn{1}{r}{$-$3.20} & \multicolumn{1}{r}{$-$1.53} \\
\multicolumn{1}{c}{68}  & \multicolumn{1}{r}{$-$3.29} & \multicolumn{1}{r}{$-$0.32} \\
\\
\\
   \hline
  \end{tabular}
%   \subcaption{潮流計算結果}
  \end{minipage}
  }
\end{table}







\begin{table}[h]
\medskip
\caption{\textbf{負荷のインピーダンス}} \label{table:ieee68loads}
 \centering
  {
  \begin{minipage}{0.32\linewidth}
    \centering
  \begin{tabular}{crrcc}
   \hline
$i$ & $r_{{\rm load}i}$  &  $x_{{\rm load}i}$ \\
   \hline \hline
\multicolumn{1}{c}{17}  & \multicolumn{1}{r}{$-$60.00} & \multicolumn{1}{r}{$-$3.00} \\
\multicolumn{1}{c}{18}  & \multicolumn{1}{r}{$-$24.70} & \multicolumn{1}{r}{$-$1.23}  \\
\multicolumn{1}{c}{19}  & \multicolumn{1}{r}{0} & \multicolumn{1}{r}{0}  \\
\multicolumn{1}{c}{20}  & \multicolumn{1}{r}{$-$6.80} & \multicolumn{1}{r}{$-$1.03}  \\
\multicolumn{1}{c}{21}  & \multicolumn{1}{r}{$-$2.74} & \multicolumn{1}{r}{$-$1.15}  \\
\multicolumn{1}{c}{22}  & \multicolumn{1}{r}{0} & \multicolumn{1}{r}{0}  \\
\multicolumn{1}{c}{23}  & \multicolumn{1}{r}{$-$2.48} & \multicolumn{1}{r}{$-$0.85} \\
\multicolumn{1}{c}{24}  & \multicolumn{1}{r}{$-$3.09} & \multicolumn{1}{r}{0.92} \\
\multicolumn{1}{c}{25}  & \multicolumn{1}{r}{$-$2.24} & \multicolumn{1}{r}{$-$0.47} \\
\multicolumn{1}{c}{26}  & \multicolumn{1}{r}{$-$1.39} & \multicolumn{1}{r}{$-$0.17}  \\
\multicolumn{1}{c}{27}  & \multicolumn{1}{r}{$-$2.81} & \multicolumn{1}{r}{$-$0.76} \\
\multicolumn{1}{c}{28}  & \multicolumn{1}{r}{$-$2.06} & \multicolumn{1}{r}{$-$0.28}  \\
\multicolumn{1}{c}{29}  & \multicolumn{1}{r}{$-$2.84} & \multicolumn{1}{r}{$-$0.27}  \\
\multicolumn{1}{c}{30}  & \multicolumn{1}{r}{0} & \multicolumn{1}{r}{0} \\
\multicolumn{1}{c}{31}  & \multicolumn{1}{r}{0} & \multicolumn{1}{r}{0} \\
\multicolumn{1}{c}{32}  & \multicolumn{1}{r}{0} & \multicolumn{1}{r}{0} \\
\multicolumn{1}{c}{33}  & \multicolumn{1}{r}{$-$1.12} & \multicolumn{1}{r}{0} \\
\multicolumn{1}{c}{34}  & \multicolumn{1}{r}{0} & \multicolumn{1}{r}{0} \\
   \hline
  \end{tabular}
%  \subcaption{データシート}
  \end{minipage}
  \begin{minipage}{0.32\linewidth}
    \centering
  \begin{tabular}{crrcc}
   \hline
$i$ & $r_{{\rm load}i}$  &  $x_{{\rm load}i}$ \\
   \hline \hline
\multicolumn{1}{c}{35}  & \multicolumn{1}{r}{0} & \multicolumn{1}{r}{0} \\
\multicolumn{1}{c}{36}  & \multicolumn{1}{r}{$-$1.02} & \multicolumn{1}{r}{0.19} \\
\multicolumn{1}{c}{37}  & \multicolumn{1}{r}{0} & \multicolumn{1}{r}{0} \\
\multicolumn{1}{c}{38}  & \multicolumn{1}{r}{0} & \multicolumn{1}{r}{0} \\
\multicolumn{1}{c}{39}  & \multicolumn{1}{r}{$-$2.67} & \multicolumn{1}{r}{$-$0.13} \\
\multicolumn{1}{c}{40}  & \multicolumn{1}{r}{$-$0.65} & \multicolumn{1}{r}{$-$0.24} \\
\multicolumn{1}{c}{41}  & \multicolumn{1}{r}{$-$10.00} & \multicolumn{1}{r}{$-$2.50} \\
\multicolumn{1}{c}{42}  & \multicolumn{1}{r}{$-$11.50} & \multicolumn{1}{r}{$-$2.50} \\
\multicolumn{1}{c}{43}  & \multicolumn{1}{r}{0} & \multicolumn{1}{r}{0} \\
\multicolumn{1}{c}{44}  & \multicolumn{1}{r}{$-$2.68} & \multicolumn{1}{r}{$-$0.05} \\
\multicolumn{1}{c}{45}  & \multicolumn{1}{r}{$-$2.08} & \multicolumn{1}{r}{$-$0.21} \\
\multicolumn{1}{c}{46}  & \multicolumn{1}{r}{$-$1.51} & \multicolumn{1}{r}{$-$0.29} \\
\multicolumn{1}{c}{47}  & \multicolumn{1}{r}{$-$2.03} & \multicolumn{1}{r}{$-$0.33} \\
\multicolumn{1}{c}{48}  & \multicolumn{1}{r}{$-$2.41} & \multicolumn{1}{r}{$-$0.02} \\
\multicolumn{1}{c}{49}  & \multicolumn{1}{r}{$-$1.64} & \multicolumn{1}{r}{$-$0.29} \\
\multicolumn{1}{c}{50}  & \multicolumn{1}{r}{$-$1.00} & \multicolumn{1}{r}{1.47} \\
\multicolumn{1}{c}{51}  & \multicolumn{1}{r}{$-$3.37} & \multicolumn{1}{r}{1.22} \\
\multicolumn{1}{c}{52}  & \multicolumn{1}{r}{$-$1.58} & \multicolumn{1}{r}{$-$0.30} \\
   \hline
  \end{tabular}
%   \subcaption{潮流計算結果}
  \end{minipage}
    \begin{minipage}{0.32\linewidth}
    \centering
  \begin{tabular}{crrcc}
   \hline
$i$ & $r_{{\rm load}i}$  &  $x_{{\rm load}i}$ \\
   \hline \hline
\multicolumn{1}{c}{53}  & \multicolumn{1}{r}{$-$2.52} & \multicolumn{1}{r}{$-$1.19} \\
\multicolumn{1}{c}{54}  & \multicolumn{1}{r}{0} & \multicolumn{1}{r}{0} \\
\multicolumn{1}{c}{55}  & \multicolumn{1}{r}{$-$3.22} & \multicolumn{1}{r}{$-$0.02} \\
\multicolumn{1}{c}{56}  & \multicolumn{1}{r}{$-$2.00} & \multicolumn{1}{r}{$-$0.74} \\
\multicolumn{1}{c}{57}  & \multicolumn{1}{r}{0} & \multicolumn{1}{r}{0} \\
\multicolumn{1}{c}{58}  & \multicolumn{1}{r}{0} & \multicolumn{1}{r}{0} \\
\multicolumn{1}{c}{59}  & \multicolumn{1}{r}{$-$2.34} & \multicolumn{1}{r}{$-$0.84} \\
\multicolumn{1}{c}{60}  & \multicolumn{1}{r}{$-$2.09} & \multicolumn{1}{r}{$-$0.71} \\
\multicolumn{1}{c}{61}  & \multicolumn{1}{r}{$-$1.04} & \multicolumn{1}{r}{$-$1.25} \\
\multicolumn{1}{c}{62}  & \multicolumn{1}{r}{0} & \multicolumn{1}{r}{0} \\
\multicolumn{1}{c}{63}  & \multicolumn{1}{r}{0} & \multicolumn{1}{r}{0} \\
\multicolumn{1}{c}{64}  & \multicolumn{1}{r}{$-$0.09} & \multicolumn{1}{r}{$-$0.88} \\
\multicolumn{1}{c}{65}  & \multicolumn{1}{r}{0} & \multicolumn{1}{r}{0} \\
\multicolumn{1}{c}{66}  & \multicolumn{1}{r}{0} & \multicolumn{1}{r}{0} \\
\multicolumn{1}{c}{67}  & \multicolumn{1}{r}{$-$3.20} & \multicolumn{1}{r}{$-$1.53} \\
\multicolumn{1}{c}{68}  & \multicolumn{1}{r}{$-$3.29} & \multicolumn{1}{r}{$-$0.32} \\
\\
\\
   \hline
  \end{tabular}
%   \subcaption{潮流計算結果}
  \end{minipage}
  }
\end{table}

\newpage
%\printindex
%
%
\end{document}