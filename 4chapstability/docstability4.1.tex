\documentclass[tombow,dvipdfmx]{corona-a5}
% dvipdfmxを追加(川口)

% Springer document settings
\usepackage[bottom]{footmisc}% places footnotes at page bottom

\usepackage{newtxtext}       % 
\usepackage[varvw]{newtxmath}       % selects Times Roman as basic font
%%%%%%%%%%%%%%%%%%%%%%%%%%%%%%%

% \usepackage{amssymb}
\usepackage{amsmath}
\usepackage{enumitem}


\usepackage{graphicx}
\usepackage{color}
\usepackage{cite}
\usepackage{makeidx}
\usepackage{ntheorem}

\usepackage{ascmac}
\usepackage{eclbkbox}
\usepackage{dsfont}

\usepackage{longtable}

\usepackage{url}

\usepackage{hyperref}

\usepackage{multicol}

%% --川口追加--
\makeatletter
\let\MYcaption\@makecaption
\makeatother
\usepackage{subcaption}
\captionsetup{compatibility=false}      % 必要に応じて

\makeatletter
\let\@makecaption\MYcaption
\makeatother
% ----

%%
\theoremstyle{plain}
\theoremheaderfont{\bfseries}
\theorembodyfont{\rmfamily}
\theoremseparator{\hspace{1ex}}
\theoremindent0cm
\theoremnumbering{arabic}
\theoremprework{\vspace{1ex}\begin{shadebox}\vspace{1ex}}
\theorempostwork{\vspace{-1ex}\end{shadebox}\vspace{1ex}}

%%
\theoremclass{theorem}

%%
\theoremclass{theorem}

%%
\theoremclass{theorem}


%%
\theoremstyle{break}
\theoremheaderfont{\bfseries}
\theorembodyfont{\rmfamily}
\theoremseparator{}
\theoremindent0cm
\theoremnumbering{arabic}
\theoremprework{\vspace{1.5ex}\begin{breakbox}\vspace{-0.5ex}}
\theorempostwork{\vspace{-0.5ex}\end{breakbox}\vspace{1.5ex}}

%%
\theoremstyle{nonumberplain}
\theoremseparator{\hspace{1ex}}

%%
\newtheorem{assumption}{Assumption}[section]

%%
\renewcommand{\theproblem}{}

\renewcommand{\theremark}{}


\newcommand{\red}[1]{{\color{red}#1}}
\newcommand{\blue}[1]{{\color{blue}#1}}
\newcommand{\green}[1]{{\color{green}#1}}

\DeclareMathOperator*{\argmax}{arg\,max}

\newcommand{\bm}[1]{\boldsymbol{#1}}
\newcommand{\sfT}{\mathsf{T}}

\newcommand{\advanced}{$^{\ddag}$}

\DeclareMathOperator{\sfsin}{\mathsf{sin}}
\DeclareMathOperator{\sfcos}{\mathsf{cos}}
\DeclareMathOperator{\sftan}{\mathsf{tan}}
\DeclareMathOperator{\sfarctan}{\mathsf{arctan}}

\DeclareMathOperator{\sfdiag}{\mathsf{diag}}
\DeclareMathOperator{\sfcol}{\mathsf{col}}
\DeclareMathOperator{\sfdet}{\mathsf{det}}
\DeclareMathOperator{\sfadj}{\mathsf{adj}}
\DeclareMathOperator{\sftrace}{\mathsf{trace}}

\DeclareMathOperator{\real}{\mathsf{Re}}

\DeclareMathOperator{\sfker}{\mathsf{ker}}
\DeclareMathOperator{\sfim}{\mathsf{im}}

\DeclareMathOperator{\sfdim}{\mathsf{dim}}
\DeclareMathOperator{\sfspan}{\mathsf{span}}

\DeclareMathOperator{\sfint}{\mathsf{int}}

\DeclareMathOperator*{\sfmin}{\mathsf{min}}
\DeclareMathOperator*{\sfmax}{\mathsf{max}}
\DeclareMathOperator*{\sfsup}{\mathsf{sup}}

\DeclareMathOperator{\sfsat}{\mathsf{sat}}

\newcommand{\mat}[1]{\left[\: \begin{matrix} #1 \end{matrix} \:\right]}
\newcommand{\spliteq}[1]{\begin{split} #1 \end{split}}
\newcommand{\simode}[1]{\begin{cases}  \begin{split} #1 \end{split} \end{cases}}

\newcommand{\proofend}{\hfill \rule{2mm}{3mm}}

\newcommand{\Xti}{X_i'}
\newcommand{\Xsi}{X_i}

\newcommand{\Xtone}{X_1'}
\newcommand{\XtN}{X_N'}

\newcommand{\Xt}{X'}
\newcommand{\Xs}{X}

\newcommand{\taudi}{\tau_i}
\newcommand{\taud}{\tau}

\newcommand{\Cgi}{b_i}


\newcommand{\Ifd}{I_{\rm field} }

\newcommand{\matlab}{\textsc{Matlab} }





%% --川口追加--
\newcommand{\thshift}{\theta_{12}}
\newcommand{\thshiftb}{\theta_{32}}
\newcommand{\Ysa}{\bm y_{12}}
\newcommand{\bca}{c_{12}}
\newcommand{\Ysb}{\bm y_{32}}
\newcommand{\bcb}{c_{32}}
\newcommand{\bcij}{c_{ij}}
\newcommand{\Is}{{\bm I}_{12}' }
\newcommand{\im}{\bm j}
\newcommand{\tr}{{\sf T}}

\begin{document}

\chapter{電力系統モデルの安定性解析}\label{sec:staana}

チャプター概要

\section{近似線形化に基づく安定性解析}\label{sec:stalin}

\subsection{クロン縮約された電力系統モデルの近似線形化}

本節では,第\ref{sec:allgen}節で議論した発電機バスがクロン縮約された等価な常微分方程式系モデルに対して,定常的な潮流状態における近似線形モデルを導出する。
まず,$i \in \mathcal{I}_{\rm G}$に関して,常微分方程式系モデルは
\begin{align}\label{eq:krondyn_}
\simode{
\dot{\delta}_i&= \omega_0  \Delta \omega_i\\
M_i   \Delta \dot{\omega}_i&= %\textstyle
 - D_i \Delta\omega_i   
 - f_i \left( \delta_i,E_i \right)_{i \in \mathcal{I}_{\rm G} }
+P_{{\rm mech}i}
\\
\tau_{{\rm d}i} \dot{E}_i & = %\textstyle
 -  \frac{X_{{\rm d}i}}{ X_{{\rm q}i} }  E_i  + \left(
X_{{\rm d}i} - X_{{\rm q}i}
\right)
g_i \left( \delta_i,E_i \right)_{i \in \mathcal{I}_{\rm G} }
+ V_{{\rm field}i}
}
\end{align}
と得られていた。
ただし,以降の議論のため,非線形項を
\begin{align}\label{eq:figi}
\spliteq{
f_i \left( \delta_i,E_i \right)_{i \in \mathcal{I}_{\rm G} } &:=
-E_i \sum_{j=1}^{|\mathcal{I}_{\rm G}|}
 E_j 
\bigl(
B_{ij}^{\rm red}
\sfsin \delta_{ij}
-
G_{ij}^{\rm red}
\sfcos \delta_{ij}
\bigr) \\
g_i \left( \delta_i,E_i \right)_{i \in \mathcal{I}_{\rm G} } &:=
-
\sum_{j=1}^{|\mathcal{I}_{\rm G}|}
E_j \bigl(
B_{ij}^{\rm red}
\sfcos \delta_{ij}
+
G_{ij}^{\rm red}
\sfsin \delta_{ij}
\bigr)
}
\end{align}
と表している。
また,$\delta_{ij}:= \delta_i - \delta_j$とした。
これらの非線形関数の各変数に関する偏微分を求めるために
\begin{align}\label{eq:defkh}
\spliteq{
k_{ij}(\delta_{ij}) & :=
-B_{ij}^{\rm red}
\sfcos \delta_{ij}
-
G_{ij}^{\rm red}
\sfsin \delta_{ij},
\\
h_{ij}(\delta_{ij}) &:= 
-B_{ij}^{\rm red}
\sfsin \delta_{ij} 
+
G_{ij}^{\rm red}
\sfcos \delta_{ij}
}
\end{align}
を定義する。
このとき,$f_i$に対して,
\begin{align}
\spliteq{
\frac{\partial f_i}{\partial \delta_i} &= 
E_i \sum_{j=1,j\neq i}^{|\mathcal{I}_{\rm G}|} E_j k_{ij}(\delta_{ij}), \\
\frac{\partial f_i}{\partial \delta_j} &=
- E_i  E_j k_{ij}(\delta_{ij}),
}
\quad
\spliteq{
\frac{\partial f_i}{\partial E_i} &=
2E_i h_{ii}(\delta_{ii})   +
 \sum_{j=1,j\neq i}^{|\mathcal{I}_{\rm G}|}
 E_j h_{ij}(\delta_{ij}), \\
 \frac{\partial f_i}{\partial E_j} &=
 E_i h_{ij}(\delta_{ij})
 }
\end{align}
が得られる。
ただし,$j \neq i$とする。
同様に,$g_i$に対して
\begin{align}
\spliteq{
\frac{\partial g_i}{\partial \delta_i} &= 
- \sum_{j=1,j\neq i}^{|\mathcal{I}_{\rm G}|} E_j h_{ij}(\delta_{ij}), 
\\
\frac{\partial g_i}{\partial \delta_j} &=
E_j h_{ij}(\delta_{ij}),
}
\quad
\spliteq{
\frac{\partial g_i}{\partial E_i} &=
k_{ii}(\delta_{ii}) , 
\\
 \frac{\partial g_i}{\partial E_j} &=
k_{ij}(\delta_{ij})
}
\end{align}
が得られる。

式\ref{eq:krondyn_}の微分方程式系に対して,発電機$i$の機械的トルク,界磁電圧,回転子偏角,内部電圧の定常値を$P_{{\rm mech}i}^{\star}$,$V_{{\rm field}i}^{\star}$,$\delta_{i}^{\star}$,$E^{\star}_i$と表す。
また,すべての$i \in \mathcal{I}_{\rm G}$に対してそれらの値を並べたベクトルを添え字$i$を除いた記号で表す。
例えば,$P_{{\rm mech}}^{\star}:=(P_{{\rm mech}i}^{\star})_{i \in \mathcal{I}_{\rm G} }$である。
これらの定常値に対して,$i \in \mathcal{I}_{\rm G}$に関する連立方程式として
\begin{align}\label{eq:kronss}
\simode{
0 &= %\textstyle
 - f_i \left( \delta_i^{\star} , E_i^{\star}  \right)_{i \in \mathcal{I}_{\rm G} }
+P_{{\rm mech}i}^{\star}
\\
0& = %\textstyle
 -  \frac{X_{{\rm d}i}}{ X_{{\rm q}i} }  E_i^{\star}  + \left(
X_{{\rm d}i} - X_{{\rm q}i}
\right)
g_i \left( \delta_i^{\star} ,E_i^{\star} \right)_{i \in \mathcal{I}_{\rm G} }
+ V_{{\rm field}i}^{\star}
}
\end{align}
が成り立つものとする。
ここで,式\ref{eq:krondyn_}における周波数偏差$\Delta \omega$の定常値は0であることを仮定していることに注意されたい。
すなわち,発電機群に対する外部入力の定常値
$(P_{{\rm mech}}^{\star},V_{{\rm field}}^{\star})$
は,需給バランスが達成される適切な値に設定されていることを仮定している。
この定常状態を基準として近似線形化を行うと,近似線形モデルが
\begin{align}\label{eq:lindyn}
\mat{
\dot{\delta}^{\rm lin} \\
M \Delta \dot{\omega}^{\rm lin} \\
\tau_{{\rm d}} \dot{E}^{\rm lin}
}
 =
\mat{
0 & \omega_0 I & 0\\
 -L & -D & -C \\
 B & 0 & A
 }
\mat{
\delta^{\rm lin} \\
\Delta \omega^{\rm lin} \\
 E^{\rm lin}
}
+
\mat{
0 & 0 \\
I & 0 \\
0 & I \\
}
\mat{
P_{{\rm mech}}^{\rm lin} \\
V_{{\rm field}}^{\rm lin}
}
\end{align}
と得られる。
ただし,添字「$\rm{lin}$」を付した状態変数および入力変数は,対応する変数の定常値との微小偏差を並べたベクトルである。
また,
\[
M:=\sfdiag \left(M_i\right)_{i \in \mathcal{I}_{\rm G} }, \qquad
D:=\sfdiag(D_i)_{i \in \mathcal{I}_{\rm G} }, \qquad
\tau_{{\rm d}}:=\sfdiag \left( \tau_{{\rm d}i} \right)_{i \in \mathcal{I}_{\rm G} }
\]
である。
さらに,式\ref{eq:defkh}の関数$k_{ij}$および$h_{ij}$に対して,第$(i,j)$要素に
\begin{align*}
\spliteq{
\hat{L}_{ij} & := \left\{
\begin{array}{cl}
E_i^{\star} \sum_{j=1, j\neq i}^{|\mathcal{I}_{\rm G}|} E_j^{\star} k_{ij}(\delta_{ij}^{\star}), & \quad i=j \\
-E_i^{\star} E_j^{\star} k_{ij}(\delta_{ij}^{\star}), & \quad i\neq j
\end{array}
\right.  \\
\hat{A}_{ij} &:=  k_{ij}(\delta_{ij}^{\star}) \\
\hat{B}_{ij}  &:= \left\{
\begin{array}{cl}
-\sum_{j=1, j\neq i}^{|\mathcal{I}_{\rm G}|} E_j^{\star} h_{ij}(\delta_{ij}^{\star}), &\quad i=j \\
E_j^{\star} h_{ij}(\delta_{ij}^{\star}), & \quad i\neq j
\end{array}
\right. \\
\hat{C}_{ij} &:= \left\{
\begin{array}{cl}
\sum_{j=1, j\neq i}^{|\mathcal{I}_{\rm G}|} E_j^{\star} h_{ij}(\delta_{ij}^{\star}), & \quad i=j \\
E_i^{\star} h_{ij}(\delta_{ij}^{\star}), & \quad i\neq j
\end{array}
\right.
}
\end{align*}
をもつ行列を$\hat{L}$,$\hat{A}$,$\hat{B}$,$\hat{C}$とするとき,行列$L$,$A$,$B$,$C$は
\begin{align}\label{eq:sysmats}
\spliteq{
L&=\hat{L} \\
A&= \sfdiag \left( X_{{\rm d}i} -  X_{{\rm q}i} \right)_{i \in \mathcal{I}_{\rm G} } \hat{A}
- \sfdiag \left(
\frac{X_{{\rm d}i}}{X_{{\rm q}i}}
\right)_{i \in \mathcal{I}_{\rm G} }  \\
B&= \sfdiag \left( X_{{\rm d}i} -  X_{{\rm q}i} \right)_{i \in \mathcal{I}_{\rm G} } \hat{B}  \\
C&= \sfdiag \bigl( 2E_i^{\star}h_{ii}(\delta_{ii}^{\star}) \bigr)_{i \in \mathcal{I}_{\rm G} }+ \hat{C} 
}
\end{align}
で定義される。
ただし,$\delta_{ij}^{\star}:=\delta_{i}^{\star}-\delta_{j}^{\star}$とする。


%\begin{align}\label{eq:sysmats}
%\spliteq{
%L&=\hat{L},\quad
%\hat{L}_{ij} := \left\{
%\begin{array}{cl}
%E_i^{\star} \sum_{j\neq i} E_j^{\star} k_{1j}(\delta_{ij}^{\star}), & i=j \\
%-E_i^{\star} E_j^{\star} k_{ij}(\delta_{ij}^{\star}), & i\neq j
%\end{array}
%\right.  \\
%A&= \sfdiag \left( X_{{\rm d}i} -  X_{{\rm q}i} \right) \hat{A}
%- \sfdiag \left(
%\frac{X_{{\rm d}i}}{X_{{\rm q}i}}
%\right) ,\quad
%\hat{A}_{ij} :=  k_{ij}(\delta_{ij}^{\star})  \\
%B&= \sfdiag \left( X_{{\rm d}i} -  X_{{\rm q}i} \right) \hat{B} ,\quad
%\hat{B}_{ij} := \left\{
%\begin{array}{cl}
%-\sum_{j\neq i} E_j^{\star} h_{ij}(\delta_{ij}^{\star}), & i=j \\
%E_j^{\star} h_{ij}(\delta_{ij}^{\star}), & i\neq j
%\end{array}
%\right. \\
%C&= \sfdiag \bigl( 2E_i^{\star}h_{ii}(\delta_{ii}^{\star}) \bigr)+ \hat{C} ,\quad
%\hat{C}_{ij} := \left\{
%\begin{array}{cl}
%\sum_{j\neq i} E_j^{\star} h_{ij}(\delta_{ij}^{\star}), & i=j \\
%E_i^{\star} h_{ij}(\delta_{ij}^{\star}), & i\neq j
%\end{array}
%\right.
%}
%\end{align}


%\begin{align}\label{eq:sysmats}
%\spliteq{
%L&:=
%\mat{
%E_1^{\star} \sum_{j\neq 1} E_j^{\star} k_{1j}(\delta_{1j}^{\star}) & -E_1^{\star} E_2^{\star} k_{12}(\delta_{12}^{\star}) & \cdots & -E_1^{\star} E_N^{\star} k_{1N}(\delta_{1N}^{\star})\\
%  -E_2^{\star} E_1^{\star} k_{21}(\delta_{21}^{\star}) &  E_2^{\star} \sum_{j\neq 2} E_j^{\star} k_{2j}(\delta_{2j}^{\star}) & \cdots & -E_2^{\star} E_N^{\star} k_{2N}(\delta_{2N}^{\star})\\
%  \vdots & \vdots & \ddots & \vdots\\
%  -E_N^{\star} E_1^{\star} k_{N1}(\delta_{N1}^{\star}) & -E_N^{\star} E_2^{\star} k_{N2}(\delta_{N2}^{\star}) & \cdots &  E_N^{\star} \sum_{j\neq N} E_j^{\star} k_{Nj}(\delta_{Nj}^{\star})
%}\\
%A &:= 
%\sfdiag \left( X_{{\rm d}i} -  X_{{\rm q}i} \right)
%\mat{
%k_{11}(\delta_{11}^{\star}) & k_{12}(\delta_{12}^{\star}) & \cdots & k_{1N}(\delta_{1N}^{\star})\\
%k_{21}(\delta_{21}^{\star}) &  k_{22}(\delta_{22}^{\star}) & \cdots & k_{2N}(\delta_{2N}^{\star})\\
%  \vdots & \vdots & \ddots & \vdots\\
%k_{N1}(\delta_{N1}^{\star}) &  k_{N2}(\delta_{N2}^{\star}) & \cdots & k_{NN}(\delta_{NN}^{\star})
%}
%\\
%& - \sfdiag \left(
%\frac{X_{{\rm d}i}}{X_{{\rm q}i}}
%\right)
%\\
%B &:= 
%\sfdiag \left( X_{{\rm d}i} -  X_{{\rm q}i} \right) 
%\\
% & \times \mat{
%-\sum_{j\neq 1} E_j^{\star} h_{1j}(\delta_{1j}^{\star}) & E_2^{\star} h_{12}(\delta_{12}^{\star}) & \cdots & E_N^{\star} h_{1N}(\delta_{1N}^{\star})\\
%E_1^{\star} h_{21}(\delta_{21}^{\star}) &  -\sum_{j\neq 2} E_j^{\star} h_{2j}(\delta_{2j}^{\star}) & \cdots & E_N^{\star} h_{2N}(\delta_{2N}^{\star})\\
%  \vdots & \vdots & \ddots & \vdots\\
%E_1^{\star} h_{N1}(\delta_{N1}^{\star}) & E_2^{\star} h_{N2}(\delta_{22}^{\star}) & \cdots & -\sum_{j\neq N} E_j^{\star} h_{Nj}(\delta_{Nj}^{\star})
%}
%\\
%C & := 
% \sfdiag \bigl( 2E_i^{\star}h_{ii}(\delta_{ii}^{\star}) \bigr) 
%\\
%& + \mat{
%\sum_{j\neq 1} E_j^{\star} h_{1j}(\delta_{1j}^{\star}) & E_1^{\star} h_{12}(\delta_{12}^{\star}) & \cdots & E_1^{\star} h_{1N}(\delta_{1N}^{\star})\\
%E_2^{\star} h_{21}(\delta_{21}^{\star}) &   \sum_{j\neq 2} E_j^{\star} h_{2j}(\delta_{2j}^{\star}) & \cdots & E_2^{\star} h_{2N}(\delta_{2N}^{\star})\\
%  \vdots & \vdots & \ddots & \vdots\\
%E_N^{\star} h_{N1}(\delta_{N1}^{\star}) & E_N^{\star} h_{N2}(\delta_{N2}^{\star}) & \cdots & \sum_{j\neq N} E_j^{\star} h_{Nj}(\delta_{Nj}^{\star})
%}
%}
%\end{align}
%である。
%


\subsection{数値計算による近似線形モデルの安定性解析}\label{sec:numlinsta}

\subsubsection{近似線形モデルの安定性}

本節では,近似線形モデルの安定性を数値的に解析することを考える。
式\ref{eq:lindyn}の近似線形モデルが安定であるか否かは,電力系統に微小な擾乱が生じた場合に,
式\ref{eq:kronss}の連立方程式を満たす定常状態に向けて,発電機群の内部状態が復するか否かを特徴づける。
擾乱の例として,発電機の機械的トルクや界磁電圧,負荷のインピーダンス値,送電線の電流値や電圧値などが,定常状態における基準値から一時的に微小変動することが挙げられる。
電力系統工学では,このような微小変動に対する安定性は,\emph{定態安定性}や\emph{同期安定性}と呼ばれる。


式\ref{eq:lindyn}の近似線形モデルの安定性は,
発電機群の内部状態の定常値
$(\delta^{\star},E^{\star})$
と外部入力の定常値
$(P_{{\rm mech}}^{\star},V_{{\rm field}}^{\star})$
の選び方によって変化することに注意されたい。
また,送電線のアドミタンスや負荷のインピーダンスの変化は,式\ref{eq:defkh}の縮約コンダクタンス$G^{\rm red}_{ij}$と縮約サセプタンス$B^{\rm red}_{ij}$を変化させる。
したがって,近似線形モデルの安定性は,上記の様々なモデルパラメータに依存して変化する。
本節の目的は,それらのモデルパラメータの変化と近似線形モデルの安定性の関係を数値的に考察することである。

\subsubsection{システム行列の固有値による安定性の判別}

式\ref{eq:lindyn}の近似線形モデルに対して,内部状態の定常値$(\delta^{\star},E^{\star})$をパラメータとして任意の値に定めれば,式\ref{eq:sysmats}のシステム行列$(L,A,B,C)$,および,
式\ref{eq:kronss}の方程式を満たす外部入力の定常値
$(P_{{\rm mech}}^{\star},V_{{\rm field}}^{\star})$は従属的に定まる。
ここで,式\ref{eq:krondyn_}の非線形の微分方程式系モデルにおいて,すべての$i \in \mathcal{I}_{\rm G}$に対して
\[
P_{{\rm mech}i}(t)=P_{{\rm mech}i}^{\star},\qquad
V_{{\rm field}i}(t)
=
V_{{\rm field}i}^{\star},\qquad 
\forall t\geq 0
\]
と設定することを考える。
これは,式\ref{eq:lindyn}の近似線形モデルにおいて
\[
P_{{\rm mech}}^{\rm lin}(t)
=0,\qquad
V_{{\rm field}}^{\rm lin}(t)
=0
,\qquad 
\forall t\geq 0
\]
と設定することを意味する。
以下では,この前提のもとで,入力を恒等的に0と仮定した自律的な線形モデル
\begin{align}\label{eq:lindynu0}
\mat{
\dot{\delta}^{\rm lin} \\
 \Delta \dot{\omega}^{\rm lin} \\
 \dot{E}^{\rm lin}
}
 =
\underbrace{
\mat{
0 & \omega_0 I & 0\\
 -M^{-1}L & -M^{-1}D & -M^{-1}C \\
 \tau_{\rm d}^{-1} B & 0 & \tau_{\rm d}^{-1} A
 }
}_{\Psi}
\mat{
\delta^{\rm lin} \\
\Delta \omega^{\rm lin} \\
 E^{\rm lin}
}
\end{align}
の安定性を解析する。
具体的には,行列$\Psi$の固有値の実部の符号を調べることによって,この近似線形モデルの安定性を判別できる。
ただし,$\Psi$は一般に零固有値を1つもつことに注意しなければならない。
実際,式\ref{eq:sysmats}の行列$L$,$B$の構造から
\begin{align}\label{eq:LBker}
L  \mathds{1} = 0
,\qquad
 B  \mathds{1} =0
\end{align}
が成り立つ。
したがって,ベクトル$v:=\sfcol(\mathds{1},0,0)$に対して
\[
\Psi v=0
\]
が成り立つ。
これは$v$が$\Psi$の零固有値に対する固有ベクトルであることを意味する。
この零固有値を除くすべての固有値の実部が負であれば,任意の初期値に対して,式\ref{eq:lindynu0}の解軌道は
\begin{align}\label{eq:linmconv}
\lim_{t\rightarrow \infty}\delta^{\rm lin}(t)= c_0  \mathds{1},\qquad
\lim_{t\rightarrow \infty}\Delta \omega^{\rm lin}(t)=0 ,\qquad
\lim_{t\rightarrow \infty} E^{\rm lin}(t)=0
\end{align}
を満たす。
ただし,$c_0$は初期値によって定まる定数である。

なお,$c_0$はどのような値であっても解析の結果に違いはない。
その理由は,式\ref{eq:krondyn_}の微分方程式系モデルにおいて,発電機の回転子偏角$\delta_i$は,他の発電機の回転子偏角$\delta_j$との差のみが意味をもつためである。
具体的には,ある$(P_{{\rm mech}}^{\star},V_{{\rm field}}^{\star})$に対して,$(\delta^{\star},E^{\star})$が式\ref{eq:kronss}の連立方程式を満たすならば,$(\delta^{\star}+c_0 \mathds{1},E^{\star})$も同じ連立方程式を満たす。
すなわち,$\delta^{\star}$と$\delta^{\star}+c_0 \mathds{1}$は本質的に同じ定常値である。


\subsubsection{近似線形化と安定性解析の実装法}


実際に,3つの発電機で構成される電力系統モデルに対して,近似線形化に基づく安定性解析を行ってみよう。

\begin{figure}[t]
\centering
\includegraphics[width = .30\linewidth]{figs/3busex}
\caption{3つの発電機からなる電力系統モデル\red{(要変更)}}
\label{fig:3genex}
\end{figure}

\begin{figure}[t]
  \centering
  {
  \begin{minipage}{0.32\linewidth}
    \centering
    \includegraphics[width = .99\linewidth]{figs/delta}
    \subcaption{ $\delta^{\rm lin}$ }
  \end{minipage}
  \begin{minipage}{0.32\linewidth}
    \centering
    \includegraphics[width = .99\linewidth]{figs/omega}
    \subcaption{ $\Delta \omega^{\rm lin}$ }
  \end{minipage}
  \begin{minipage}{0.32\linewidth}
    \centering
    \includegraphics[width = .99\linewidth]{figs/E}
    \subcaption{ $E^{\rm lin}$ }
  \end{minipage}
  }
  \caption{近似線形モデルの初期値応答(青:発電機1,赤:発電機2,黒:発電機3)}
  \label{fig:timeex}
\end{figure}

%\begin{figure}[t!]
%\centering
%{    
%    \includegraphics[width = .5\linewidth]{figs/delta}
%    \subcaption{ $\delta^{\rm lin}$ }
%    \includegraphics[width = .5\linewidth]{figs/omega}
%    \subcaption{ $\Delta \omega^{\rm lin}$ }
%    \includegraphics[width = .5\linewidth]{figs/E}
%    \subcaption{ $E^{\rm lin}$ }
%}
%\caption{近似線形モデルの初期値応答(青:発電機1,赤:発電機2,黒:発電機3)}
%\label{fig:timeex}
%\end{figure}

\begin{figure}[t]
  \centering
  {
  \begin{minipage}{0.32\linewidth}
    \centering
    \includegraphics[width = .85\linewidth]{figs/gam01}
    \subcaption{ $\gamma=0.1$ }
  \end{minipage}
  \begin{minipage}{0.32\linewidth}
    \centering
    \includegraphics[width = .85\linewidth]{figs/gam2}
    \subcaption{ $\gamma=2$ }
  \end{minipage}
  \begin{minipage}{0.32\linewidth}
    \centering
    \includegraphics[width = .85\linewidth]{figs/gam5}
    \subcaption{ $\gamma=5$ }
  \end{minipage}
  \caption{近似線形モデルが安定となるパラメータの領域}
  \label{fig:gamsta}
  }
\end{figure}

\begin{例}[近似線形モデルの数値的な安定性解析]\label{ex:linsyssim}
\ref{fig:3genex}に示される3つの発電機で構成される電力系統モデルを考える。
以下では,発電機の物理定数は共通であるものとして,すべての$i \in \{1,2,3\}$に対して
\begin{align*}
M_i=1
,\qquad
D_i=1
,\qquad
\tau_{{\rm d}i} = 0.01
,\qquad
X_{{\rm d}i} = 1.01
,\qquad
X_{{\rm q}i} = 1
\end{align*}
に設定する。
また,系統周波数は$\omega_0=1$とする。

式\ref{eq:lindynu0}の近似線形モデルを導出するため,内部状態の定常値
$(\delta^{\star},E^{\star})$を設定する。
内部電圧の定常値は各発電機で異なる値として
\begin{subequations}\label{eq:exEsds}
\begin{align}
E^{\star}_1=1
,\qquad
E^{\star}_2=2
,\qquad
E^{\star}_3=3
\end{align}
に設定する。
回転子偏角差の定常値は,パラメータ$\theta_1 \in [0, 1]$を用いて
\begin{align}
\delta_{12}^{\star}= - \frac{\pi}{2} \theta_1
,\qquad
\delta_{13}^{\star}=  \frac{\pi}{2} \theta_1
\end{align}
と表す。
\end{subequations}
パラメータ$\theta_1$の大きさは,定常状態における回転子偏角差の大きさに対応している。
このパラメータの変化は,式\ref{eq:sysmats}のシステム行列の変化として近似線形モデルに現れる。
送電網のアドミタンス行列も同様に,2つのパラメータ$\gamma >0$と$\theta_2 \in [0,1]$を用いて
\begin{align*}
\bm{Y} =
\theta_2
\mat{
\gamma & 0 & 0 \\
0 &\gamma & 0 \\
0 & 0 & \gamma
}
 +
\bm{j} (1-\theta_2) 
\mat{
-1 & 1 & 0 \\
1 & -2 & 1 \\
0 & 1 & -1 
}
\end{align*}
と表す。
ここで,$\gamma$はアドミタンス行列の実部の絶対値を指定するパラメータであり,$\theta_2$は実部と虚部の相対的な大小関係を指定するパラメータである。
アドミタンス行列のパラメータ変化は,式\ref{eq:defkh}の縮約コンダクタンス$B^{\rm red}_{ij}$と縮約サセプタンス$G^{\rm red}_{ij}$の値の変化として近似線形モデルに現れる。
上記のパラメータ$(\gamma,\theta_1,\theta_2)$を変化させて,近似線形モデルの安定性を数値的に解析する。

まず,$\gamma=2$,$\theta_1=0.3$,$\theta_2=0.2$とした場合のシステムの初期値応答を確認してみよう。
適当な初期値を設定した場合の結果を\ref{fig:timeex}に示す。
この図から,このパラメータ設定の場合には,式\ref{eq:linmconv}のように発電機群の内部状態が収束していることがわかる。

つぎに,(a) $\gamma=0.1$,(b) $\gamma=2$,(c) $\gamma=5$に設定する。
この(a)--(c)の場合において,$\theta_1$と$\theta_2$を0.01の刻み幅で変化させて式\ref{eq:lindynu0}の$\Psi$の固有値を調べることにより,近似線形モデルが安定であるか否かを網羅的に確認してみよう。
\ref{fig:gamsta}にその結果を示す。
近似線形モデルが安定となったパラメータを青の領域で表している。
\ref{fig:gamsta}(a)--(c)から,$\gamma$が大きくなる,すなわち,アドミタンス行列の実部であるコンダクタンス行列の絶対値が大きくなるにしたがって,近似線形モデルが安定となるパラメータの領域が狭くなっていることがわかる。

また,どの図においても原点付近のパラメータ領域,すなわち,定常状態における回転子偏角差が小さく,かつ,アドミタンス行列の実部が虚部よりも相対的に小さい場合には,近似線形モデルが安定であることがわかる。
なお,$\gamma=5$,$\theta_1=1$,$\theta_2 \in [0.56,0.99]$のパラメータでは,挙動は非常に振動的であるが,例外的に安定であった。
これらの結果から,定常状態における回転子偏角差が小さく,コンダクタンス行列がサセプタンス行列よりも相対的に小さい場合に,近似線形モデルが安定になりやすい傾向にあることがわかる。
\end{例}




\subsection{近似線形モデルの数学的な安定性解析\advanced}

\subsubsection{近似線形モデルの安定性と同期}

本節では,式\ref{eq:lindynu0}の近似線形モデルの安定性を数学的に解析する。
その安定性は行列$\Psi$の固有値により特徴づけられる。
第\ref{sec:numlinsta}節で議論されているように,$\Psi$は正則ではなく,
\begin{align}\label{eq:eqset}
\mathcal{M} =
 \sfspan\left\{
 \mat{
 \mathds{1}\\
 0\\
 0
 }
 \right\}
\end{align}
は$\Psi$の零固有値に対する固有空間である。
この集合は,相対的な値を一定に保ってすべての発電機の偏角を変化させた等価な定常値の集合を表している。
したがって,近似線形モデルの状態が式\ref{eq:eqset}の平衡点集合のうちのどの点に収束するかは問題とならない。
この事実に基づき,つぎの定義を与える。

\begin{定義}[近似線形モデルの同期]
\label{def:stalin}
式\ref{eq:lindynu0}の近似線形モデルを考える。
任意の初期値に対して,内部状態が式\ref{eq:eqset}の平衡点集合$\mathcal{M}$のいずれかの点に収束するとき,近似線形モデルは\emph{同期する}と呼ぶ。
\end{定義}

定義\ref{def:stalin}における同期は,任意の初期値に対して,式\ref{eq:linmconv}が成り立つことを表している。
なお,式\ref{eq:linmconv}において$c_0$の値は任意であるため,その任意性を「$\mathcal{M}$のいずれかの点に収束すること」として表現している。

以下では,式\ref{eq:lindynu0}の$\Psi$の核空間が1次元であること,すなわち,
\[
\sfker \Psi = \mathcal{M}
\]
であることを仮定する。
これは近似線形モデルの同期が達成されるための必要条件である。
特に,$A$が正則である場合には,
\begin{align}\label{eq:nescon}
\sfker (L-CA^{-1}B) = \sfspan
\left\{
\mathds{1}
\right\}
\end{align}
と等価であることが示される。


\subsubsection{近似線形モデルのフィードバック系による表現}

式\ref{eq:lindynu0}の近似線形モデルを2つのサブシステムのネガティブ・フィードバック系として記述することを考える。
1つ目のサブシステムは,周波数偏差に関する微分方程式系として
\begin{align}\label{eq:Fss}
F: \simode{
M \Delta \dot{\omega}^{\rm lin} &= -D \Delta \omega^{\rm lin}
+
u_F \\
y_F &= \omega_0 \Delta \omega^{\rm lin}
}
\end{align}
とする。
以下では,このサブシステムを\emph{機械的サブシステム}と呼ぶ。
2つ目のサブシステムは,回転子偏角と内部電圧に関する微分方程式系として
\begin{align}\label{eq:Gss}
G: \simode{
\dot{\delta}^{\rm lin} & = u_G \\
\tau_{\rm d} \dot{E}^{\rm lin} &= A E^{\rm lin} + B \delta^{\rm lin} \\
y_G &= C E^{\rm lin} + L \delta^{\rm lin}
}
\end{align}
とする。
このサブシステムを\emph{電気的サブシステム}と呼ぶ。
これら2つのサブシステムを
\[
u_F = -y_G,\qquad
u_G = y_F
\]
のようにネガティブ・フィードバック結合することで,式\ref{eq:lindynu0}の近似線形モデルが表現される。
以降の安定性解析は,
\begin{itemize}
\item 機械的サブシステムと電気的サブシステムの伝達関数はともに正実である
\item 正実な伝達関数のネガティブ・フィードバック系は安定である
\end{itemize}
という事実に基づく。

\subsubsection{機械的サブシステムの解析}

まず,式\ref{eq:Fss}の機械的サブシステム$F$を考える。
このサブシステムの入力$u_F$から出力$y_F$までの伝達関数は
\begin{align}\label{eq:trFs}
F(s) :=  
\sfdiag \left( 
\frac{\omega_0}{M_i s + D_i}
\right)
\end{align}
である。
この伝達関数は単純な1次系が並列されたシステムであるため,その性質は容易に調べられる。
例えば,$\omega_0$,$M_i$,$D_i$はすべて正の定数であることから,$F(s)$は常に安定である。
伝達関数の安定性はつぎのように定義される。

\begin{定義}[伝達関数の安定性]\label{def:trsta}
伝達関数$Q(s)$のすべての極の実部が負であるとき,$Q(s)$は\emph{安定}であると呼ぶ。
\end{定義}

式\ref{eq:trFs}の$F(s)$の極,すなわち,分母多項式$M_i s + D_i$の零点は$s=-\frac{D_i}{M_i}$であることから,$F(s)$は安定であることがわかる。
さらに,例\ref{ex:Fspr1}で後述するように,$F(s)$は正実であることも示される。
伝達関数の正実性(positive realness)はつぎのように定義される。

\begin{定義}[伝達関数の正実性]\label{def:trpf}
正方な伝達関数$Q(s)$に対して
\begin{align}\label{eq:defOm0}
\Omega_0 := \left\{
\omega_0 \in \mathbb{R}: 
\mbox{ 純虚数$\bm{j} \omega_0$が$Q(s)$の極である}
\right\}
\end{align}
を定義する。
つぎの3条件が満たされるとき,$Q(s)$は\emph{正実}であると呼ぶ。
\begin{itemize}
\item $Q(s)$のすべての極の実部は非正である。
\item すべての$\omega \in [0,\infty)\setminus \Omega_0$に対して,$Q(\bm{j} \omega) + Q^{\sf T}(-\bm{j} \omega)$が半正定である。
\item 純虚数の極が存在するとき,それらの重複度は1であり,かつ,留数に対して
\begin{align*}
\lim_{s \rightarrow \bm{j} \omega_0} (s-\bm{j} \omega_0) Q(s) = \lim_{s\rightarrow \bm{j} \omega_0} \{ (s-\bm{j} \omega_0) Q(s)\}^{\sf *}\succeq 0
,\qquad
\forall \omega_0 \in \Omega_0
\end{align*}
が成り立つ。
\end{itemize}
\end{定義}

定義\ref{def:trpf}において,特に重要なものは1つ目と2つ目の条件である。
1つ目の条件は伝達関数の安定性を表している。
ただし,極の実部が0である場合も許容されている。
2つ目の条件は,伝達関数を虚軸上で評価したときの複素対称な部分の正定性に関するものである。
特に,$Q(s)$がスカラーの場合,すなわち,入力と出力がどちらもスカラーである場合には,2つ目の条件はすべての$\omega \in [0,\infty)\setminus \Omega_0$に対して$Q(\bm{j}\omega)$の実部が非負であることを表している。
ただし,$Q(s)$が行列である場合には,一般に
\begin{align*}
Q(\bm{j} \omega) + Q^{\sf T}(-\bm{j} \omega) \neq 2 \real\left[ Q(\bm{j} \omega) \right]
\end{align*}
であることに注意されたい。
なお,実数係数をもつ$Q(s)$に対しては$Q^{\sf T}( -\bm{j} \omega)$は$\{Q(\bm{j} \omega)\}^*$に等しい。
3つ目の条件は,$Q(s)$が純虚数の極をもつ場合の例外的なものであるが,例えば,積分器$\frac{1}{s}$などのように原点に極をもつ伝達関数を解析するために必要となる。

なお,式\ref{eq:trFs}の伝達関数$F(s)$の正実性は,つぎのように確かめられる。

\begin{例}[並列された安定な1次系の正実性]\label{ex:Fspr1}
式\ref{eq:trFs}の$F(s)$が正実であることを確認してみよう。
まず,$F(s)$のすべての極の実部は負であるため,1つ目の条件は満たされている。
また,純虚数の極は存在しないため,3つ目の条件を考慮する必要はない。
さいごに,2つ目の条件について
\begin{align*}
\sfdiag \left( 
\frac{\omega_0}{ \bm{j} \omega M_i + D_i}
\right)
+
\left\{
\sfdiag \left( 
\frac{\omega_0}{\bm{j} \omega M_i + D_i}
\right)
\right\}^*
=
\sfdiag \left( 
\frac{2 \omega_0 D_i}{\omega^2 M_i^2 + D_i^2}
\right)
\end{align*}
であることから,すべての$\omega\in [0,\infty)$に対してその半正定性が示される。
なお,$\Omega_0=\emptyset$である。
\end{例}



\subsubsection{電気的サブシステムの解析}


つぎに,式\ref{eq:Gss}の電気的サブシステム$G$を考える。
このサブシステムの入力$u_G$から出力$y_G$までの伝達関数は
\begin{align}\label{eq:trGs}
G(s) :=  - \frac{1}{s} 
\underbrace{
\left\{ -C \bigl( \tau_{\rm d} s -A \bigr)^{-1} B - L \right\}
}_{H(s)}
\end{align}
である。
以下では,この$G(s)$が正実であるための必要十分条件が,式\ref{eq:figi}における縮約コンダクタンス
に対して
\begin{align}\label{eq:Gredcon}
G^{\rm red}_{ij}=0,\qquad 
\forall i\neq j
\end{align}
であることを示す。
その準備として,正実性と対になる概念である伝達関数の負虚性(negative imaginaryness)を導入する\cite{petersen2010feedback,xiong2010negative}。

\begin{定義}[伝達関数の負虚性]
\label{def:trni}
原点に極をもたない正方な伝達関数$Q(s)$に対して,式\ref{eq:defOm0}の$\Omega_0$を定義する。
つぎの3条件が満たされるとき,$Q(s)$は\emph{負虚}であると呼ぶ。
\begin{itemize}
\item $Q(s)$のすべての極の実部は非正である。
\item すべての$\omega \in (0,\infty)\setminus \Omega_0$に対して,$\bm{j}\left\{Q(\bm{j} \omega) - Q^{\sf T}(-\bm{j} \omega) \right\}$が半正定である。
\item 純虚数の極が存在するとき,それらの重複度は1であり,かつ,留数に対して
\begin{align*}
\lim_{s \rightarrow \bm{j} \omega_0} (s-\bm{j} \omega_0) \bm{j} Q(s) = 
\lim_{s\rightarrow \bm{j} \omega_0} \{ (s-\bm{j} \omega_0) \bm{j} Q(s)\}^{\sf *}\succeq 0
,\quad
\forall \omega_0 \in \Omega_0
\end{align*}
が成り立つ。
\end{itemize}
\end{定義}

定義\ref{def:trpf}の正実性が,伝達関数の複素対称な部分に関する半正定性により定義されていたのに対して,定義\ref{def:trni}の負虚性は,伝達関数の複素歪対称な部分に関する半正定性により定義されている。
特に,$Q(s)$がスカラーの場合には,
\begin{align*}
\bm{j}\left\{Q(\bm{j} \omega) - Q^{\sf T}(-\bm{j} \omega) \right\}
= 2 \imag[Q(\bm{j} \omega)]
\end{align*}
であることから,2つ目の条件はすべての$\omega \in (0,\infty)\setminus \Omega_0$に対して$Q(\bm{j}\omega)$の虚部が非負であることを表している。
なお,定義\ref{def:trni}では,定義\ref{def:trpf}との対比のため$Q(s)$が虚軸上に極をもつことを許容しているが,以下の議論では安定な伝達関数に対して負虚性を考えるため,2つ目の条件のみに注目すれば良い。


%式\ref{eq:sysmats}のシステム行列に現れた関数$k_{ij}(\delta_{ij}^{\star})$や$h_{ij}(\delta_{ij}^{\star})$には,式\ref{eq:defGam}で定義される$\bm{\mathit{\Gamma}}$の逆行列の実部や虚部が混在しており,解析的な扱いは容易ではない。
%一方で,アドミタンス行列$\bm{Y}$の実部が0であるとき,$\bm{\mathit{\Gamma}}$は実対称かつ正定な行列となるため,式\ref{eq:sysmats}のシステム行列にも対称性や歪対称性などの扱いやすい構造が現れる。
%
%一般に,アドミタンス行列の非対角要素の実部はすべて非負であることから,定理\ref{thm:PQ}より,アドミタンス行列の実部が0であることが,系統全体での有効電力の送電損失が0となるための必要十分条件であることがわかる。
%これらの事実に基づき,つぎの定義を導入する。
%
%\begin{定義}[無損失な送電網]\label{def:lless}
%式\ref{eq:Ypig}のアドミタンス行列$\bm{Y}$の実部が0であるとき,送電網は\emph{無損失}であると呼ぶ。
%\end{定義}


以上の定義に基づいて,式\ref{eq:trGs}の$G(s)$に対してつぎの事実が示される。

\begin{定理}[電気的サブシステムの伝達関数の正実性]
\label{thm:EdynNI}
式\ref{eq:trGs}の伝達関数$H(s)$は安定であると仮定する。
このとき,$H(s)$が負虚であることは,式\ref{eq:Gredcon}が成り立つことと等価である。
また,式\ref{eq:trGs}の伝達関数$G(s)$が正実であることは,
式\ref{eq:Gredcon}が成り立ち,かつ
\begin{align}\label{eq:pdsp}
L-CA^{-1}B = (L-CA^{-1}B)^{\sf T} \succeq 0
\end{align}
が成り立つことと等価である。
\end{定理}

\begin{証明}
まず,$H(s)$が負虚であるならば,式\ref{eq:Gredcon}が成り立つことを示す。
このためには,その対偶として,式\ref{eq:Gredcon}が成り立たないならば,$H(s)$が負虚ではないことを示せば良い。
これに対して,式\ref{eq:Gredcon}が成り立たないとき,$L$は非対称であることがわかる。
また,
\[
\lim_{\omega \rightarrow \infty} \bm{j}
\left\{
H(\bm{j}\omega)-H^{\sf T}(-\bm{j}\omega)
\right\}
=\bm{j}
\left(
-L+L^{\sf T}
\right)
\]
であることから,$L$が対称でないならば$H(s)$は負虚ではない。
したがって,式\ref{eq:Gredcon}が成り立たないならば,$H(s)$は負虚ではないことが示される。


つぎに,式\ref{eq:Gredcon}が成り立つならば,$H(s)$は負虚であることを示す。
補題\ref{lem:nilem}より,$H(s)$が負虚であることを示すためには,$L$が対称であり,かつ
\begin{align}\label{eq:cndQ}
\tilde{A}^{\sf T}Q + Q\tilde{A} \preceq 0
,\qquad
Q\tilde{A}^{-1}\tilde{B}=C^{\sf T}
\end{align}
を満たす正定な行列$Q$が存在することを示せば良い。
ただし,
\begin{align*}
\tilde{A}:= \tau_{\rm d}^{-1}A
,\qquad
\tilde{B}:= \tau_{\rm d}^{-1}B
\end{align*}
とする。
ここで,式\ref{eq:Gredcon}が成り立つならば,
\begin{align*}
k_{ij}(\delta_{ij}^{\star}) =
k_{ji}(\delta_{ji}^{\star})
,\qquad
h_{ij}(\delta_{ij}^{\star}) = 
- h_{ji}(\delta_{ji}^{\star}),\qquad
h_{ii}(\delta_{ii}^{\star}) = 0
\end{align*}
が成り立つことから,$L$は対称である。
また,$H(s)$は安定であることから
\begin{align*}
\tilde{A} = 
\sfdiag \left( \frac{X_{{\rm d}i} -  X_{{\rm q}i}}{\tau_{{\rm d}i}} \right)
\tilde{A}'
\end{align*}
において,$X_{{\rm d}i} > X_{{\rm q}i}$であることから
\[
\tilde{A}':=
\mat{
k_{11}(\delta_{11}^{\star}) & k_{12}(\delta_{12}^{\star}) & \cdots & k_{1N}(\delta_{1N}^{\star})\\
k_{21}(\delta_{21}^{\star}) &  k_{22}(\delta_{22}^{\star}) & \cdots & k_{2N}(\delta_{2N}^{\star})\\
  \vdots & \vdots & \ddots & \vdots\\
k_{N1}(\delta_{N1}^{\star}) &  k_{N2}(\delta_{N2}^{\star}) & \cdots & k_{NN}(\delta_{NN}^{\star})
}
- \sfdiag \left(
\frac{X_{{\rm d}i}}{X_{{\rm q}i}(X_{{\rm d}i}-X_{{\rm q}i})}
\right)
\]
は負定である。
したがって,$Q=-\tilde{A}'$は式\ref{eq:cndQ}を満たす正定行列となることから,$H(s)$が負虚であることが示される。


つぎに,$G(s)$に関する等価性を示す。
いま,$H(s)$は安定であることを仮定していることから,$G(s)$の虚軸上の極は原点のみであり,その重複度は1である。
したがって,$G(s)$が
正実であるための必要十分条件は,
\begin{align}\label{eq:Gpr}
G(\bm{j}\omega) + G^{\sf T}(-\bm{j}\omega) \succeq 0
,\qquad \forall \omega \in \mathbb{R}\setminus\{0\}
\end{align}
が成り立つこと,かつ
\begin{align}\label{eq:cndG0}
\lim_{s\rightarrow 0} s G(s) = \lim_{s\rightarrow 0} \{ s G(s)\}^{\sf T}\succeq 0
\end{align}
が成り立つことである。
式\ref{eq:Gredcon}が成り立つとき,式\ref{eq:Gpr}が成り立つことは
\begin{align}\label{eq:NIineq}
G(\bm{j}\omega) + G^{\sf T}(-\bm{j}\omega)
=
\frac{\bm{j}}{\omega} \left\{
H(\bm{j}\omega) - H^{\sf T}(-\bm{j}\omega)
\right\}
,\qquad \forall \omega \in \mathbb{R}\setminus\{0\}
\end{align}
より,$H(s)$が負虚であることから示される。
また,
\begin{align*}
\lim_{s\rightarrow 0} s G(s) =
L - C\tilde{A}^{-1}\tilde{B} = L - C A^{-1} B
\end{align*}
であることから,式\ref{eq:cndG0}の半正定性は式\ref{eq:pdsp}の条件に等しい。
なお,式\ref{eq:Gredcon}が成り立つとき,$L$は対称であり,
\begin{align*}
C\tilde{A}^{-1}\tilde{B} = C Q^{-1} C^{\sf T}
\end{align*}
も対称であることから,式\ref{eq:cndG0}の対称性も示される。

逆に,式\ref{eq:Gredcon}が成り立たない,または,式\ref{eq:pdsp}の条件が満たされないならば$G(s)$が正実でないことを示す。
後者については,式\ref{eq:pdsp}の条件が式\ref{eq:cndG0}の条件に等しいことから明らかである。
また,式\ref{eq:Gredcon}が成り立たないとき,$H(s)$は負虚ではないことから,
\begin{align*}
\lambda_{\rm min}\left[
\bm{j}
\left\{
H(\bm{j}(\omega_0 + \alpha )) - H^{\sf T}(-\bm{j}(\omega_0 + \alpha ))
\right\}
\right] < 0
,\qquad
\forall \alpha \in (0,\epsilon] 
\end{align*}
となるある点$\omega_0\geq 0$および十分に小さな$\epsilon >0$が存在する。
ただし,$\lambda_{\rm min}$は最小固有値を表す。
したがって,0ではないすべての$\omega \in (\omega_0, \omega_0+\epsilon] $に対して
$G(\bm{j} \omega) + G^{\sf T}(-\bm{j} \omega) $は半正定ではない。
\end{証明}



\subsubsection{偏差サブシステムの漸近安定性}

式\ref{eq:lindynu0}の近似線形モデルが同期することの必要十分条件は,行列$\Psi$の零固有値を除くすべての固有値の実部が負であることである。
以下の議論のため,つぎの2つの用語を定義しておく。

\begin{定義}[正方行列の安定性]
\label{def:matsta}
正方行列$A$に対して,そのすべての固有値の実部が負であるとき,$A$は\emph{安定}であると呼ぶ。
\end{定義}

\begin{定義}[線形システムの漸近安定性]
\label{def:difsta}
行列$A$が安定であるとき,線形システム
\[
\dot{x}=Ax
\]
は\emph{漸近安定}であると呼ぶ。
\end{定義}

式\ref{eq:lindynu0}の行列$\Psi$は零固有値をもつため,定義\ref{def:matsta}の意味で安定ではないことに注意されたい。
式\ref{eq:lindyn}の近似線形モデルの同期を解析するために,零固有値に対応する不変な固有空間を取り除くことを考えよう。
具体的には,式\ref{eq:lindyn}の$3N$次元の微分方程式から1次元の不変部分空間を除いた$(3N-1)$次元のサブシステムを求める。
この$(3N-1)$次元のサブシステムを\emph{偏差サブシステム}と呼ぶ。
以下では,偏差サブシステムが漸近安定であることが,式\ref{eq:lindyn}の近似線形モデルが同期することの必要十分条件となることを確認する。

式\ref{eq:lindyn}の状態$\delta^{\rm lin}$に対して基底変換を適用することにより,偏差サブシステムを導出することを考える。
このために,$\delta^{\rm lin}$の不変な成分に対応する,式\ref{eq:eqset}の固有空間$\mathcal{M}$は,式\ref{eq:LBker}に示されている$L$と$B$の共通の核空間に由来することに注目する。
この共通の核空間が存在することは,
\begin{align}\label{eq:sfkerW}
\sfker W = \sfspan\{ \mathds{1}\}
\end{align}
を満たす任意の行列$W \in \mathbb{R}^{(N-1)\times N}$を用いて
\begin{align}\label{eq:decLB}
L = L_0 W 
,\qquad
B = B_0 W 
\end{align}
と分解できることを意味している。
この事実を具体例で確認してみよう。

\begin{例}[同一の核空間をもつ行列による分解]\label{ex:Ldec}
式\ref{eq:sysmats}の$L$について,$N=3$である場合を考えてみよう。
このとき,$L$は
\begin{align*}
L=
\mat{
L_{12}+L_{13} & -L_{12} & -L_{13}\\
-L_{21}& L_{21} + L_{23} & -L_{23}\\
- L_{31}& -L_{32} & L_{31}+L_{32}
}
\end{align*}
と表せる。
ただし,$L_{ij}:= E_i^* E_j^* k_{ij}(\delta_{ij}^*)$である。
これに対して,式\ref{eq:sfkerW}を満たす行列,すなわち,$W \mathds{1}=0$を満たす行フルランクな行列の例として
\begin{align*}
W=\mat{
1 & -1 & 0 \\
0 & 1 & -1
}
\end{align*}
を考える。
この$W$を用いる場合には,$L$は
\begin{align*}
L = \underbrace{
\mat{
L_{12} + L_{13} & L_{13} \\
-L_{21}& L_{23} \\
-L_{31} & -(L_{31}+L_{32})
}
}_{L_0}
W
\end{align*}
のように分解できる。
一般に,$W$の$WW^{\dagger}=I$となる適当な右逆行列$W^{\dagger}$を用いれば,$L_0$は$LW^{\dagger}$として求められる。
この例の場合では,その右逆行列は
\begin{align*}
W^{\dagger} = \mat{
1 & 1\\
0 & 1\\
0 & 0
}
,\qquad
W^{\dagger} = \mat{
1 & 2\\
0 & 2\\
0 & 1
}
\end{align*}
などのように選ぶことができる。
このように,$W^{\dagger}$は一意ではないが,$WW^{\dagger}=I$であれば,いかなる$W^{\dagger}$に対しても結果は等しい。
なお,式\ref{eq:sysmats}の$B$も$L$と同様の構造をもつため,共通の$W$を用いて分解することが可能である。
\end{例}

式\ref{eq:sfkerW}を満たす適当な行列$W \in \mathbb{R}^{(N-1)\times N}$を用いて,式\ref{eq:lindyn}の$\delta^{\rm lin}$に関する基底変換として
\begin{align}\label{eq:delbt}
\delta^{\rm lin}_{\rm e}  := W \delta^{\rm lin}
,\qquad
\overline{\delta}^{\rm lin}_{\rm e} :=
\frac{1}{N} \mathds{1}^{\sf T}
\delta^{\rm lin}
\end{align}
を定義する。
このとき
\begin{align*}
\mat{
W \\
\frac{1}{N} \mathds{1}^{\sf T}
}^{-1}
= \mat{
W^{\dagger}& \mathds{1}
}
\end{align*}
が満たされるように$W^{\dagger}$を定義することができる。
これは
\begin{align*}
\mat{
W \\
\frac{1}{N} \mathds{1}^{\sf T}
}
\mat{
W^{\dagger}& \mathds{1}
}
=\mat{
WW^{\dagger} & W \mathds{1}\\
\frac{1}{N} \mathds{1}^{\sf T} W^{\dagger} & \frac{1}{N} \mathds{1}^{\sf T} \mathds{1}
}
=\mat{
I & 0\\
0 & 1
}
\end{align*}
において$\frac{1}{N} \mathds{1}^{\sf T}W^{\dagger}=0$であるような$W^{\dagger}$を選ぶことに等しい。
このように定義すると,式\ref{eq:delbt}の基底変換に対する逆変換は
\begin{align}\label{eq:batrinv}
\delta^{\rm lin}
=
W^{\dagger}
\delta^{\rm lin}_{\rm e} +
\mathds{1}
\overline{\delta}^{\rm lin}_{\rm e}
\end{align}
と表せる。
ここで,$\mathds{1}$は$\delta^{\rm lin}$の共通成分を表す基底であり,$W^{\dagger}$はそれ以外の偏差成分を表す基底である。
すなわち,$\delta^{\rm lin}$の共通成分が$\overline{\delta}^{\rm lin}_{\rm e}$であり,偏差成分が$\delta^{\rm lin}_{\rm e}$である。
この基底変換を式\ref{eq:lindyn}の近似線形モデルに行うと
\begin{align}\label{eq:lindynnew}
\mat{
\dot{\overline{\delta}}^{\rm lin}_{\rm e} \\
\dot{\delta}^{\rm lin}_{\rm e} \\
M \Delta \dot{\omega}^{\rm lin} \\
\tau_{{\rm d}} \dot{E}^{\rm lin}
}
&=
\mat{
0 & 0 & \frac{\omega_0}{N} \mathds{1}^{\sf T} & 0 \\
0& 0 & \omega_0 W & 0\\
0&  -L_0 & -D & -C \\
0 & B_0 & 0 & A
 }
\mat{
\overline{\delta}^{\rm lin}_{\rm e} \\
\delta^{\rm lin}_{\rm e} \\
\Delta \omega^{\rm lin}\\
 E^{\rm lin}
}
+
\mat{
0 & 0 \\
0 & 0 \\
I & 0 \\
0 & I \\
}
\mat{
P_{{\rm mech}}^{\rm lin} \\
V_{{\rm field}}^{\rm lin}
}
\end{align}
が得られる。
ただし,$L_0 = L W^{\dagger}$,$B_0 = B W^{\dagger}$である。
この表現において,1つ目の状態として現れているスカラの状態変数$\overline{\delta}^{\rm lin}_{\rm e}$に注目しよう。
この状態変数は,その他の状態変数である
$(\delta^{\rm lin}_{\rm e},\Delta \omega^{\rm lin}, E^{\rm lin})$には影響を及ぼさない。
したがって,その冗長な状態変数$\overline{\delta}^{\rm lin}_{\rm e}$を削除して得られる
\begin{align}\label{eq:lindynred}
\mat{
\dot{\delta}^{\rm lin}_{\rm e} \\
M \Delta \dot{\omega}^{\rm lin} \\
\tau_{{\rm d}} \dot{E}^{\rm lin}
}
&=
\mat{
 0 & \omega_0 W & 0\\
  -L_0 & -D & -C \\
 B_0 & 0 & A
 }
\mat{
\delta^{\rm lin}_{\rm e} \\
\Delta \omega^{\rm lin}\\
 E^{\rm lin}
}
+
\mat{
0 & 0 \\
I & 0 \\
0 & I \\
}
\mat{
P_{{\rm mech}}^{\rm lin} \\
V_{{\rm field}}^{\rm lin}
}
\end{align}
は,式\ref{eq:lindynnew}の状態変数$(\delta^{\rm lin}_{\rm e},\Delta \omega^{\rm lin}, E^{\rm lin})$の挙動を厳密に再現する低次元なサブシステムとなる。
これが$(3N-1)$次元の偏差サブシステムを記述する状態方程式である。

式\ref{eq:lindynred}の偏差サブシステムが漸近安定であること,すなわち,任意の初期値に対して
\[
\lim_{t\rightarrow \infty}\delta^{\rm lin}_{\rm e}(t)= 0 ,\qquad
\lim_{t\rightarrow \infty}\Delta \omega^{\rm lin}(t)=0 ,\qquad
\lim_{t\rightarrow \infty} E^{\rm lin}(t)=0
\]
が成り立つことが示されれば,式\ref{eq:batrinv}の関係から,式\ref{eq:linmconv}が成り立つことが示される。
ただし,
\[
c_0=
\lim_{t\rightarrow \infty} \overline{\delta}^{\rm lin}_{\rm e}(t) 
\]
である。
以上の議論から,式\ref{eq:lindynred}の偏差サブシステムが漸近安定であることは,式\ref{eq:lindyn}の近似線形モデルが同期することの必要十分条件であることがわかる。


\subsubsection{偏差サブシステムの漸近安定性}



,すわなち,
\begin{align*}
\mat{
I &0&0 \\
0&\sfdiag(M_i) & 0 \\
0&0 &\sfdiag(\tau_{{\rm d}i}) 
}^{-1}
\mat{
 0 & \omega_0 W & 0\\
  -L_0 & -\sfdiag(D_i) & -C \\
 B_0 & 0 & A
 }
\end{align*}
が安定であることを示すためには,式\ref{eq:trFs}の$F(s)$と式\ref{eq:trGseq}右辺の$G(s)$から構成される$(3N-1)$次元のフィードバック系が内部安定であることを示せば良い。
特に,式\ref{eq:trGseq}の右辺と左辺は$s$の関数としては等しいことから,左辺の伝達関数が正実であるならば右辺の伝達関数も正実である。
以上の事実から,$G(s)$の表現形式にかからわずその正実性が示されるのであれば,式\ref{eq:lindyn}の線形近似システムが同期することを結論づけられる。
なお,式\ref{eq:lindynu0}の$\Psi$の核空間が1次元,すなわち,式\ref{eq:eqset}の$\mathcal{M}$に等しい限りは,$G(s)$の内部で2つ以上の積分器が相殺されることはない。

%%%%%%%%%%%%%



式\ref{eq:decLB}の分解を用いて,ラプラス領域における\eqref{eq:trGs}の$G(s)$の変形として
\begin{align}\label{eq:trGseq}
\spliteq{
& - \frac{1}{s} I \cdot
\underbrace{
\left\{ -C \bigl( \sfdiag(\tau_{{\rm d}i})s -A \bigr)^{-1} B - L \right\}
}_{H(s)}
\\
& \hspace{8em}=
- 
\underbrace{
\left\{ -C \bigl( \sfdiag(\tau_{{\rm d}i})s -A \bigr)^{-1} B_0 - L_0 \right\} 
}_{H_0(s)}
\cdot
\frac{1}{s} W
}
\end{align}
を考える。
ラプラス領域においてこの等式は「自明な」ものであるが,時間領域における$G(s)$の実現を考える場合には異なる解釈が与えられる。
具体的には,$G(s)$への入力を$u_g(s)$とすると,式\ref{eq:trGseq}の左辺は,$u_g(s)$を伝達関数$H(s)$に作用させて出てきた信号を$\frac{1}{s}$で積分するものと解釈できる。
一方で,右辺は$W u_g(s)$という信号を$\frac{1}{s}$で積分した後で$H_0(s)$に作用させるものと解釈できる。

式\ref{eq:trGseq}の変形で重要な点は,「積分器の数が両辺で異なる」という事実である。
具体的には,左辺では$H(s) u_g(s)$という$N$次元の信号を各要素で並列に積分するため,積分器は$N$個存在するのに対して,右辺では$W u_g(s)$という$(N-1)$次元の信号の積分であるため,積分器は$(N-1)$個しか存在していない。
これは,右辺の表現では$G(s)$の入出力関係に影響しない冗長な積分器が削除されていることを意味している。
この冗長な積分器は$L$や$B$の共通の核空間に由来するものであることに注意されたい。


\begin{figure}[t]
\centering
\includegraphics[width = .75\linewidth]{figs/blocklinsysnew}
\caption{冗長な積分器が取り除かれた偏差サブシステムのブロック線図表現}
\label{fig:blocklinnew}
\end{figure}

式\ref{eq:trGseq}右辺の表現によれば,\ref{fig:blocklin}のフィードバック系を\ref{fig:blocklinnew}のように変形することができる。
この図の右上に位置する$\delta^{\rm lin}_{\rm e}$は,$N$次元の$\delta^{\rm lin}$から不変な固有空間の成分を削除した$(N-1)$次元の偏差成分を表している。
より正確にはつぎのように説明できる。






\subsubsection{近似線形モデルの伝達関数表現}

システム制御理論における伝達関数の正実性やそれに類する概念に基づいて,式\ref{eq:lindyn}の近似線形モデルの同期を解析することを考えよう。
まず,時間領域における式\ref{eq:lindyn}の微分方程式をラプラス変換して,ラプラス領域(周波数領域)におけるシステム表現を導出する。
\ref{fig:blocklin}は式\ref{eq:lindyn}をラプラス変換して得られるブロック線図である。
この図は以下の手順により導出できる。
いま,3行目の$E^{\rm lin}$に関する微分方程式に注目すると,そのラプラス変換は
\begin{align*}
s \sfdiag(\tau_{{\rm d}i}) E^{\rm lin}(s)
= B \delta^{\rm lin}(s) + A E^{\rm lin}(s) + V^{\rm lin}_{\rm field}(s)
\end{align*}
と得られる。
ただし,ラプラス領域の変数も時間領域の変数と同じ記号を用いて表している。
これを$E^{\rm lin}(s)$に関して解けば
\begin{align*}
E^{\rm lin}(s) = \bigl( \sfdiag(\tau_{{\rm d}i})s -A \bigr)^{-1} 
\left\{ B \delta^{\rm lin}(s)
+ V^{\rm lin}_{\rm field}(s) \right\}
\end{align*}
が得られる。
これは\ref{fig:blocklin}の最下段にある右と中央のブロックに相当する。
同様に,式\ref{eq:lindyn}の1行目と2行目の微分方程式もラプラス変換し,注目している変数に関する方程式を代数的に解くことによって,\ref{fig:blocklin}のその他のブロックも得られる。
なお,\ref{fig:blocklin}左側に位置する$P^{\rm lin}$は,各発電機バスに供給される有効電力を並べたベクトルであると解釈できる。

\begin{figure}[t]
\centering
\includegraphics[width = .75\linewidth]{figs/blocklinsys2}
\caption{伝達関数に基づく近似線形モデルのブロック線図表現}
\label{fig:blocklin}
\end{figure}




つぎに,\ref{fig:blocklin}のブロック線図を解析しやすい形式に変形することを考える。
具体的には,右上の$\frac{\omega_0}{s}I$のブロックを分割し,左上のブロックと$\omega_0$の積として



を定義する。
これにより,\ref{fig:blocklin}から入力を除いたブロック線図は,\ref{fig:staFG}(a)のように2つのシステムのネガティブ・フィードバック結合として表現できる。

\subsubsection{伝達関数の安定性と正実性}



この伝達関数の極は,時間領域における状態空間実現\red{(付録??)}である



に対するシステム行列の固有値と一致する。
なお,$F(s)$の微分方程式による実現は,座標変換の自由度があるため唯一ではないが,その固有値は実現に依らず不変である。

さらに,


\begin{figure}[t]
  \centering
  {
  \begin{minipage}{0.49\linewidth}
    \centering
    \includegraphics[width = .99\linewidth]{figs/staFG}
    \subcaption{ }
  \end{minipage}
  \begin{minipage}{0.49\linewidth}
    \centering
    \includegraphics[width = .99\linewidth]{figs/staFGIO}
    \subcaption{ }
  \end{minipage}
  \caption{ネガティブ・フィードバック系}
  \label{fig:staFG}
  }
\end{figure}



\subsubsection{正実性に基づくフィードバック系の安定性解析}

以下では,伝達関数の正実性が,\ref{fig:staFG}(a)のようなフィードバック系の安定性を示すために有用な性質であることを説明する。
ただし,その準備として,伝達関数で表現されたフィードバック系に対する安定性を新たに定義する必要がある。
その理由は,\ref{fig:staFG}(a)では,$F(s)$や$G(s)$の入出力信号はフィードバック結合に使われてしまっており,外部から印加する入力信号や観測する出力信号が存在しないためである。
このことは,\ref{fig:staFG}(a)のフィードバック系には,入力と出力の関係を表す「伝達関数」そのものが定義されないことを意味する。
したがって,伝達関数に基づくフィードバック系の安定性解析に,定義\ref{def:trsta}の安定性の定義をそのまま用いることはできない。
以下の議論では,フィードバック系に対してつぎの安定性の定義を用いる。


\begin{定義}[フィードバック系の内部安定性]\label{def:fbsta}
\ref{fig:staFG}(a)のネガティブ・フィードバック系に対して,\ref{fig:staFG}(b)に示されるように入力と出力を追加することを考える。
このとき,$(u_F,u_G)$から$(y_F,y_G)$までの伝達関数
%\begin{align}\label{eq:defTs}
%\spliteq{
%& \underbrace{
%\mat{
%T_{y_Fu_F}(s)& T_{y_Fu_G}(s)\\
%T_{y_Gu_F}(s)& T_{y_G u_G}(s)
%}
%}_{T(s)}
%:= 
%\\
%& \mat{
%\bigl( I+ F(s)G(s) \bigr)^{-1}F(s) & -\bigl( I+ F(s)G(s) \bigr)^{-1}F(s) G(s) \\
%\bigl( I+ G(s)F(s) \bigr)^{-1}G(s)F(s) & \bigl( I+ G(s)F(s) \bigr)^{-1} G(s)
%}
%}
%\end{align}
\begin{align}\label{eq:defTs}
T(s)
:=\mat{
\bigl( I+ F(s)G(s) \bigr)^{-1}F(s) & -\bigl( I+ F(s)G(s) \bigr)^{-1}F(s) G(s) \\
\bigl( I+ G(s)F(s) \bigr)^{-1}G(s)F(s) & \bigl( I+ G(s)F(s) \bigr)^{-1} G(s)
}
\end{align}
に対して,$T(s)$を構成する4つの伝達関数がすべて安定であるとき,\ref{fig:staFG}(a)のネガティブ・フィードバック系は\emph{内部安定}であると呼ぶ。
\end{定義}

定義\ref{def:fbsta}における4つの伝達関数の安定性は,\ref{fig:staFG}(a)のフィードバック系を時間領域の微分方程式で表現した場合に,そのシステムが漸近安定となることと等価であることが知られている。
したがって,定義\ref{def:fbsta}は一見すると煩雑であるが,素朴に直感する時間領域における安定性と同義であると解釈すれば良い。
なお,式\ref{eq:trFs}の$F(s)$の定義から,\ref{fig:staFG}(b)における$u_F$と$y_F$の信号は,\ref{fig:blocklin}における$P_{\rm mech}^{\rm lin}$と$\omega_0 \Delta \omega^{\rm lin}$にそれぞれ対応することがわかる。
また,$y_G$は\ref{fig:blocklin}の$P^{\rm lin}$に対応する。
一方で,$u_G$は式\ref{eq:lindyn}の微分方程式には存在せず,伝達関数に基づくフィードバック系の内部安定性を定義するために現れた「仮想的な」入力であることに注意されたい。


\begin{figure}[t]
  \centering
  {
  \begin{minipage}{0.49\linewidth}
    \centering
    \includegraphics[width = .90\linewidth]{figs/nyquistFG}
    \subcaption{ $F(s)$と$G(s)$のナイキスト線図 }
  \end{minipage}
  \begin{minipage}{0.49\linewidth}
    \centering
    \includegraphics[width = .90\linewidth]{figs/nyquistFGop}
    \subcaption{  積$F(s)G(s)$のナイキスト線図 }
  \end{minipage}
  \caption{正実な伝達関数とそれらの積のナイキスト線図(赤十字は$-1 + 0 \bm{j}$の点を表す)}
  \label{fig:nyquistpr}
  }
\end{figure}


\begin{例}[正実な伝達関数のフィードバック系の安定性解析]
\label{ex:nyquistpr}
正実な伝達関数から構成されるネガティブ・フィードバック系の内部安定性をナイキストの安定判別法\red{(付録??)}の観点で考察してみよう。
説明の簡単化のため,\ref{fig:staFG}(a)における$F(s)$と$G(s)$はスカラーの伝達関数であるとする。
具体的には
\begin{align*}
F(s)=\frac{2}{s+1}
,\qquad
G(s)=\frac{3(12s+1)}{2(3s+1)(s+1)}
\end{align*}
と設定する。
このとき,$F(s)$と$G(s)$が正実な伝達関数であることは,\ref{fig:nyquistpr}(a)に示されるように,それらのナイキスト線図が複素平面上の虚軸より右側に存在することを意味している。
すなわち,$F(s)$と$G(s)$の周波数応答関数を
\begin{align*}
F(\bm{j} \omega) = |F(\bm{j} \omega)| e^{\bm{j} \angle F(\bm{j} \omega)}
,\qquad
G(\bm{j} \omega) = |G(\bm{j} \omega)| e^{\bm{j} \angle G(\bm{j} \omega)}
\end{align*}
と極座標で表示したときに,それらの位相$\angle F(\bm{j} \omega)$と$\angle G(\bm{j} \omega)$はどちらも,すべての$\omega$に対して$\left[ -\frac{\pi}{2},\frac{\pi}{2} \right]$の範囲に収まっていることを表している。
したがって,それらの積の周波数応答関数
\begin{align*}
F(\bm{j} \omega) G(\bm{j} \omega)
=
|F(\bm{j} \omega)| |G(\bm{j} \omega)| e^{\bm{j} \left\{\angle F(\bm{j} \omega) + \angle G(\bm{j} \omega)\right\}}
\end{align*}
の位相は,すべての$\omega$に対して$[-\pi,\pi]$の範囲にあることが導かれる。
この「位相が$[-\pi,\pi]$の範囲にある」という事実は,一見すると複素平面上のすべての位相を表しており特別な意味がないように見えるが,重要な点は「積の周波数応答関数のナイキスト線図は,負の実軸を横断するような軌跡は描かないこと」を表していることである。
この様子を\ref{fig:nyquistpr}(b)に示している。
したがって,$F(s)$と$G(s)$の積のナイキスト線図が,それらのネガティブ・フィードバック系の内部安定性を特徴づける$-1 + 0 \bm{j}$の点を右に見ながら周回しないことが結論づけられる。

しかしながら,$F(s)$と$G(s)$の正実性だけでは,積のナイキスト線図が$-1 + 0 \bm{j}$の点の上を「ちょうど通過する」ような安定限界となる場合が含まれている。
実応用においては,このような状況,すなわち
\begin{align*}
F(\bm{j} \omega_0) G(\bm{j} \omega_0) = -1 + 0 \bm{j}
\end{align*}
となる$\omega_0$が存在するような状況は,$F(s)$と$G(s)$を意図的に選ばない限りはほとんど起こらないが,数学的に厳密にフィードバック系の内部安定性を示すためには,正実性よりも少しだけ強い条件を$F(s)$や$G(s)$に課す必要がある。
この「少しだけ強い条件」の与え方には様々なものが考えられる。
例えば,すべての$\omega \in [0,\infty)$に対して,$F(\bm{j} \omega)$の位相が$ \left[-\frac{\pi}{2},\frac{\pi}{2} \right]$の境界には触れないこと,すなわち,$\left(-\frac{\pi}{2},\frac{\pi}{2} \right)$の範囲にあることがわかるだけで,$F(s)$と$G(s)$のネガティブ・フィードバック系の内部安定性が数学的に証明できる。
\end{例}


例\ref{ex:nyquistpr}で示されているように,伝達関数の正実性はフィードバック系の内部安定性を議論するために有用である。
しかしながら,注目する伝達関数が式\ref{eq:trFs}の$F(s)$のように単純なものではない場合には,その伝達関数が正実かどうかを解析的に判断することは難しい。




\begin{例}[並列された安定な1次系に対する正実補題]\label{ex:Fspr2}
補題\ref{lem:prlem}を用いて式\ref{eq:trFs}の$F(s)$の正実性を確認してみよう。
式\ref{eq:Fss}は$F(s)$の時間領域における実現であることから,補題\ref{lem:prlem}の行列に当てはめて考えると
\begin{align*}
A = -\sfdiag \left( 
\frac{D_i}{ M_i} 
\right)
,\qquad 
B= \sfdiag \left( 
\frac{1}{ M_i} 
\right)
,\qquad
C= \omega_0 I 
,\qquad
D=0
\end{align*}
となる。
このとき,$P=\omega_0 \sfdiag(M_i)$と選べば,式\ref{eq:prlem}の行列不等式に対して
\begin{align*}
\mat{
-2 \sfdiag \left( 
\omega_0 D_i
\right)
 & 0 \\
0 & 0
}\preceq 0
\end{align*}
が得られるため,$F(s)$が正実であることが結論づけられる。
なお,この例のように,出力に入力が直接的に現れないシステム,すなわち,$D=0$であるシステムについては,その伝達関数の正実性は
\begin{align}\label{eq:Dzero}
A^{\sf T}P+PA \preceq 0
,\qquad 
PB=C^{\sf T}
\end{align}
を満たす正定な$P$が存在することと等価である。
ただし,$(A,B)$は可制御とする。


式\ref{eq:trFs}に対する座標変換として,$\tilde{x}_f := \omega_0 x_f$を考えてみよう。
このとき,$F(s)$の別の実現として
\begin{align*}
F: \simode{
\dot{\tilde{x}}_f &= \textstyle - \sfdiag \left( 
\frac{D_i}{ M_i} 
\right)
\tilde{x}_f
+ 
\omega_0 \sfdiag \left( 
\frac{1}{ M_i} 
\right)
 u_f \\
y_f &=  \tilde{x}_f
}
\end{align*}
が得られる。
この実現に対しては,$P=\omega_0^{-1} \sfdiag(M_i)$が式\ref{eq:prlem}の行列不等式の解として得られることがわかる。
このように,式\ref{eq:prlem}の行列不等式の解は,その状態空間実現に依存して変化することに注意されたい。
\end{例}

式\ref{eq:Fss}の$F$のように,入力が出力に直接的に現れないシステムの場合,すなわち,出力が状態変数の線形結合のみで記述される場合,その伝達関数の分母多項式の次数は分子多項式の次数よりも厳密に大きい。
このような伝達関数は,\emph{厳密にプロパー}な伝達関数と呼ばれる。
\ref{fig:staFG}(a)において,$F(s)$と$G(s)$の両方が厳密にプロパーでない場合には,それらのフィードバック系が定義されない場合があることに注意されたい。
これは,$F(s)$への入力信号が瞬時的にその出力信号として現れ,その出力信号が$G(s)$を通して瞬時的に$F(s)$にフィードバックされるような,入出力信号の瞬時的な無限ループが生じ得るためである。
なお,$F(s)$と$G(s)$の少なくともどちらか一方が厳密にプロパーであればそのような問題は生じないため,以下では必要に応じて伝達関数が厳密にプロパーであることを条件として課す。

つぎの補題が示すように,2つの正実な伝達関数によるネガティブ・フィードバック系は正実性をもつ。


\begin{補題}[正実な伝達関数のフィードバック系]\label{lem:prpre}
\ref{fig:staFG}(b)において,伝達関数$F(s)$と$G(s)$はどちらも正実であるとする。
ただし,$F(s)$と$G(s)$の少なくともどちらか一方は厳密にプロパーであるとする。
このとき,式\ref{eq:defTs}の$T(s)$によって定義される$(u_F,u_G)$から$(y_F,y_G)$までの伝達関数は正実である。
\end{補題}

\begin{証明}
時間領域において,$F(s)$と$G(s)$のネガティブ・フィードバック系は
\begin{align*}
F: \simode{
\dot{x}_F &= A_F x_F + B_F u_F\\
y_F &= C_F x_F + D_F u_F ,
}
\qquad
G: \simode{
\dot{x}_G &= A_G x_G + B_G y_F\\
u_F &= -C_G x_G 
}
\end{align*}
と表せる。
ただし,$G(s)$が厳密にプロパーである場合を考えているが,$F(s)$が厳密にプロパーである場合も同様である。
このとき,フィードバック系の状態方程式をひとつにまとめると
\begin{align*}
\mat{
\dot{x}_F \\ \dot{x}_G
}
 =
 \underbrace{
\mat{
A_F & -B_F C_G\\
B_G C_F & A_G - B_G D_F C_G
}
}_{A_T}
\mat{x_F \\ x_G}
+
\underbrace{
\mat{
B_F & 0\\
B_G D_F & B_G
}
}_{B_T}
\mat{u_F \\ u_G}
\end{align*}
が得られる。
同様に,出力方程式は
\begin{align*}
\mat{
y_F \\ y_G
}
 =
\underbrace{
\mat{
C_F & -D_F C_G\\
0 & C_G
}
}_{C_T}
\mat{x_F \\ x_G}
+ 
\underbrace{
\mat{
D_F & 0\\
0 & 0
}
}_{D_T}
\mat{u_F \\ u_G}
\end{align*}
とまとめられる。
定義より,$T(s) = C_T (sI -A_T)^{-1}B_T + D_T$である。
ここで,$F(s)$と$G(s)$の正実性の仮定から
\begin{align}\label{eq:asumpr}
\mat{
A_F^{\sf T}P_F+P_F A_F & P_FB_F-C_F^{\sf T} \\
B_F^{\sf T} P_F -C_F & -(D_F+D_F^{\sf T})
}\preceq 0
,\qquad
\simode{
&A_G^{\sf T}P_G+P_GA_G \preceq 0\\
&P_GB_G=C_G^{\sf T}
}
\end{align}
を満たす正定な$P_F$と$P_G$が存在する。
また,$P_F$と$P_G$をブロック対角に並べた行列を$P_T$とすると
\begin{align*}
A_T^{\sf T}P_T+P_T A_T 
&=
\mat{
A_F^{\sf T}P_F+P_F A_F & (C_F^{\sf T} -P_F B_F) C_G \\
C_G^{\sf T} (C_F^{\sf T} -P_F B_F)^{\sf T} & 
A_G^{\sf T}P_G+P_GA_G - C^{\sf T}_G (D_F + D_F^{\sf T})C_G
},
\\
P_T B_T &= \mat{
P_F B_F & 0 \\
C_G^{\sf T} D_F & C_G^{\sf T}
}
\end{align*}
が得られる。
ただし,式\ref{eq:asumpr}右の等式を用いた。
したがって
\begin{align}\label{eq:augpr}
\mat{
A_T^{\sf T}P_T+P_T A_T & P_T B_T - C_T^{\sf T} \\
B_T^{\sf T} P_T -C_T & -(D_T+D_T^{\sf T})
}
=
\mat{
X + Y & 0\\
0 & 0
}
\end{align}
となる。
ただし
\begin{align*}
X&:= 
\mat{
A_F^{\sf T}P_F+P_F A_F & -(P_FB_F-C_F^{\sf T})C_G & P_FB_F-C_F^{\sf T} \\
-C_G^{\sf T}(P_FB_F-C_F^{\sf T})^{\sf T} & - C^{\sf T}_G (D_F + D_F^{\sf T})C_G & C^{\sf T}_G (D_F + D_F^{\sf T})\\
(P_FB_F-C_F^{\sf T})^{\sf T} & (D_F + D_F^{\sf T})C_G & -(D_F + D_F^{\sf T})
},
\\
Y &:= 
\mat{
0&0&0\\
0&A_G^{\sf T}P_G+P_GA_G&0\\
0&0&0
}
\end{align*}
である。
式\ref{eq:asumpr}右の不等式より,$Y$は半負定であることから,式\ref{eq:augpr}の半不定性を示すためには,$X$が半不定であることを示せば良い。
実際,$X$は
\begin{align*}
X=
\mat{
I & 0 \\
0 & -C_G^{\sf T} \\
0 & I
}
\mat{
A_F^{\sf T}P_F+P_F A_F & P_FB_F-C_F^{\sf T} \\
B_F^{\sf T} P_F -C_F & -(D_F+D_F^{\sf T})
}
\mat{
I & 0 \\
0 & -C_G^{\sf T} \\
0 & I
}^{\sf T}
\end{align*}
と書き換えられることから,式\ref{eq:asumpr}左の不等式によって,式\ref{eq:augpr}の半不定性が示される。
これは$F(s)$と$G(s)$のフィードバック系が正実であることを意味している。
\end{証明}

補題\ref{lem:prpre}に示されている正実性が保存される性質は,式\ref{eq:lindyn}の近似線形モデルを伝達関数に基づき解析する場合にも有用である。
例\ref{ex:Fspr1}や例\ref{ex:Fspr2}で示されているように,式\ref{eq:trFs}の$F(s)$は正実である。
このとき,式\ref{eq:trGs}の$G(s)$の正実性が示されるのであれば,それらのネガティブ・フィードバック系は正実となることから,少なくとも式\ref{eq:lindyn}の内部状態は発散しないことが保証できる。
一方で,$F(s)$と$G(s)$が正実であることだけでは,フィードバック系の内部安定性を数学的に保証することはできない。
この内部安定性を示すために,つぎの事実が利用できる。

\begin{補題}[正定行列フィードバックによる正実な伝達関数の安定化]\label{lem:posfb}
\ref{fig:staFG}(a)において,伝達関数$F(s)$は正実かつ厳密にプロパーであり,$G(s)$は対称行列$G_0$であるとする。
このとき,$G_0$が正定であるならば,そのネガティブ・フィードバック系は内部安定である。
\end{補題}

\begin{証明}
\red{ナイキストで言ったほうが早いかも?}
時間領域において,$F(s)$と$G(s)$のネガティブ・フィードバック系は
\begin{align*}
F: \simode{
\dot{x}_F &= A x_F + B u_F\\
y_F &= C x_F ,
}
\qquad
G: u_F = - G_0 y_F
\end{align*}
と表せる。
ただし,$F$は最小実現であるものとする。
この表現から,$u_F$と$y_F$を代入により消去すると,フィードバック系の簡潔な表現として
\begin{align*}
\dot{x}_F = (A -B G_0 C) x_F
\end{align*}
が得られる。
正実補題から,式\ref{eq:Dzero}を満たすある正定な$P$が存在することに注意して,
\red{リアプノフ関数}の候補として$V(x_F):= x_F^{\sf T} P x_F$と選ぶと
\begin{align*}
\spliteq{
\frac{d}{dt} V(x_F) &= \dot{x}_F^{\sf T} P x_F + x_F^{\sf T} P \dot{x}_F\\
&= x_F^{\sf T} (\underbrace{A^{\sf T}P +PA - 2 C^{\sf T}G_0 C}_{M})x_F 
}
\end{align*}
が得られる。
\red{ここで,$(C,A)$は可観測であることから$M$は負定である。}
\end{証明}





\begin{例}[正実性に基づくフィードバック系の安定性解析]\label{ex:entsys}
補題\ref{lem:prpre}と補題\ref{lem:posfb}を組み合わせることにより,\ref{fig:staFG}(a)のフィードバック系の内部安定性を解析してみよう。
ただし,以下では,式\ref{eq:trGs}の$G(s)$が正実であることを仮定して議論する。
まず,式\ref{eq:Fss}の$F(s)$の状態空間実現について,それ自身が正実な伝達関数と正定な行列のネガティブ・フィードバック系であるとみなす。
具体的には,$F$を2つに分解することにより
\begin{align*}
F_0 &:\simode{
\dot{x}_f &= \textstyle - \sfdiag \left( 
\frac{D_i}{ M_i} - \kappa_i 
\right)
x_f
+ \sfdiag \left( 
\frac{1}{ M_i} 
\right)
(u_f + v_f) \\
y_f &= \omega_0 x_f,
}
\\
K&: y_k = \sfdiag \left( \frac{ M_i \kappa_i}{\omega_0} \right) y_f
,\qquad
v_f = - y_k
\end{align*}
とする。
仮想的な入出力である$v_f$と$y_k$を代入により消去すれば,この$F_0$と$K$のフィードバック系が式\ref{eq:Fss}に一致することは容易にわかる。
この分解は\ref{fig:Fdec}において点線のブロックで示されている。
また,$\kappa_i \in \left(0,\frac{D_i}{M_i} \right)$である適当な$\kappa_i$を選べば$F_0$は正実であり,$K$は正定行列による静的なフィードバックを表すことがわかる。
一方で,$G(s)$の正実性を仮定していることから,$F_0(s)$と$G(s)$を
\begin{align*}
F_0&:\simode{
\dot{x}_f &= \textstyle - \sfdiag \left( 
\frac{D_i}{ M_i} - \kappa_i 
\right)
x_f
+ \sfdiag \left( 
\frac{1}{ M_i} 
\right)
(u_f + v_f)
\\
y_f &= \omega_0 x_f,
}
\\
G&: \simode{
\dot{x}_g &= A_g x_g + B_g y_f\\
y_g &= C_g x_g, 
}
\qquad
u_f = -y_g
\end{align*}
のようにフィードバック結合した場合において,$v_f$から$y_f$までの伝達関数は正実となる。
したがって,\ref{fig:Fdec}のように,全体系は$F_0(s)$と$G(s)$から構成される正実な伝達関数に,$K$という正定行列のフィードバックを加えた系であるとみなすことができる。
これは\ref{fig:staFG}(a)のフィードバック系が内部安定であることを意味している。
\end{例}

\begin{figure}[t]
\centering
\includegraphics[width = .45\linewidth]{figs/Fdec}
\caption{正実性に基づくフィードバック系の内部安定性の保証}
\label{fig:Fdec}
\end{figure}

例\ref{ex:entsys}で示されているように,式\ref{eq:Fss}の$F(s)$が安定な1次系の並列であるという事実から,式\ref{eq:trGs}の$G(s)$が正実であれば\ref{fig:staFG}(a)のフィードバック系が内部安定であることが示される。
しかしながら,この内部安定性が,\ref{fig:blocklin}のフィードバック系の内部安定性を必ずしも意味しないことに注意されたい。
実際,式\ref{eq:eqset}が示すように,式\ref{eq:lindyn}の近似線形モデルには不変な固有空間が存在するため漸近安定ではない。
したがって,\ref{fig:blocklin}のフィードバック系は内部安定にはならない。
この相違は,\ref{fig:blocklin}を\ref{fig:staFG}(a)に変換して,積分器を含むいくつかのブロックを$G(s)$としてまとめたことにより,$G(s)$の内部で積分器の原点極が零点と相殺されていることに起因する。
次節では,この$G(s)$の内部の\red{極零相殺}にも注意して,より精密に近似線形モデルの同期を解析していく。

\subsection{正実性に基づく近似線形モデルの同期解析\advanced}\label{sec:syncanp}


\subsubsection{近似線形モデルの正実性}




定理\ref{thm:EdynNI}は,送電網が無損失であることが,式\ref{eq:trGs}の伝達関数$G(s)$が正実であることを特徴づける条件の1つであることを示している。
なお,もう1つの条件である式\ref{eq:pdsp}の意味はつぎのように説明できる。
式\ref{eq:lindyn}の近似線形モデルにおいて,内部電圧の状態方程式である
\begin{align*}
\sfdiag(\tau_{{\rm d}i})
 \dot{E}^{\rm lin} = 
A E^{\rm lin}(t) + B \delta^{\rm lin}(t) 
+ V^{\rm lin}_{\rm field}(t)
\end{align*}
に注目する。
この微分方程式において,すべての時定数$\tau_{{\rm d}i}$が0に漸近する極限を考える。
これは「内部電圧が定常状態に達するために必要な時間が,$\delta^{\rm lin}(t)$や$V^{\rm lin}_{\rm field}(t)$が変動する時間と比較して十分に短いような極限を考えること」に相当する。
例えば,$\delta^{\rm lin}(t)$と$V^{\rm lin}_{\rm field}(t)$の時間変化が非常に緩やかであり,定数$\delta^{\rm lin}_0$と$V^{\rm lin}_{{\rm field}0}$とみなせると仮定すると,$E^{\rm lin}(t)$は一定の時間が経過すれば
\begin{align*}
E^{\rm lin}_0 = -A^{-1} B \delta^{\rm lin}_0
-A^{-1} V^{\rm lin}_{{\rm field}0}
\end{align*}
という値に漸近する。
ただし,$A$は安定である必要がある。
したがって,$E^{\rm lin}(t)$が定常状態に達する時間が十分に短い,すなわち,すべての時定数$\tau_{{\rm d}i}$が十分に小さい場合には,$E^{\rm lin}(t)$と比較して緩やかに変化する$\delta^{\rm lin}(t)$や$V^{\rm lin}_{\rm field}(t)$に対しても
\begin{align}\label{eq:spa}
E^{\rm lin}(t) \simeq  -A^{-1} B\delta^{\rm lin}(t)
- A^{-1} V^{\rm lin}_{\rm field}(t)
\end{align}
という近似が成り立つ。
このような状態変数の時間スケールの違いを利用して,収束が速い微分方程式を代数方程式で近似する方法は,システム制御理論において\emph{特異摂動近似}(singular perturbation approximation)と呼ばれている。
\red{実際,内部電圧の動特性は,機械的なタービンの動特性に比較して時定数が小さいため,このような特異摂動近似を介して電力系統モデルの挙動を解析することは有意義である。}
式\ref{eq:spa}が等号として成り立つものとして$\Delta \omega^{\rm lin}(t)$の微分方程式に代入すれば
\begin{align}\label{eq:spamodel}
\spliteq{
\mat{
\dot{\hat{\delta}}^{\rm lin} \\
\sfdiag(M_i) \Delta \dot{\hat{\omega}}^{\rm lin} 
}
&=
\mat{
 0 & \omega_0 I \\
  -(L-CA^{-1}B) & -\sfdiag(D_i)  \\
 }
\mat{
\hat{\delta}^{\rm lin} \\
\Delta \hat{\omega}^{\rm lin}
}
\\
&+
\mat{
0 & 0 \\
I & CA^{-1}
}
\mat{
P_{{\rm mech}}^{\rm lin} \\
V_{{\rm field}}^{\rm lin}
}
}
\end{align}
が得られる。
ただし,近似的な動特性であることを表すため状態変数を
$\hat{\delta}^{\rm lin}$,
$\Delta \hat{\omega}^{\rm lin}$として区別した。
また,内部電圧の近似的な値は
\begin{align*}
\hat{E}^{\rm lin}:=  -A^{-1} B \hat{\delta}^{\rm lin}
- A^{-1} V^{\rm lin}_{\rm field}
\end{align*}
として与えられる。
式\ref{eq:spamodel}の特異摂動近似システムは,線形2次振動子の結合系となっており,式\ref{eq:pdsp}の条件は,ばね定数をまとめた行列の半正定性を保証するものであることがわかる。
また,式\ref{eq:spamodel}の平衡点集合は
\begin{align}\label{eq:Msp}
\mathcal{M}_{\rm sp} := \sfspan 
\left\{
\mat{ \mathds{1}\\0 }
\right\}
\end{align}
である。

\begin{例}[特異摂動近似システムの初期値応答]
例\ref{ex:linsyssim}で扱った3つの発電機で構成される近似線形モデルに対して,式\ref{eq:spamodel}の特異摂動近似システムの初期値応答を確認してみよう。
\ref{fig:gamsta}と同じ設定で,特異摂動近似システムの初期値応答を重ねてプロットしたものが\ref{fig:timeexsp}である。
この例では,内部電圧の時定数がすべての発電機で$\tau_{{\rm d}i}=0.01$という相対的に小さな値に設定されているため,近似線形モデルと特異摂動近似システムの状態の軌道の差はほとんど生じていないことがわかる。
また,内部状態が式\ref{eq:Msp}の平衡点集合$\mathcal{M}_{\rm sp}$の点に収束していることもわかる。
\end{例}

\begin{figure}[t]
  \centering
  {
  \begin{minipage}{0.32\linewidth}
    \centering
    \includegraphics[width = .99\linewidth]{figs/deltasp}
    \subcaption{ 実線:$\delta^{\rm lin}$,点線:$\hat{\delta}^{\rm lin}$ }
  \end{minipage}
  \begin{minipage}{0.32\linewidth}
    \centering
    \includegraphics[width = .99\linewidth]{figs/omegasp}
    \subcaption{ 実線:$\Delta \omega^{\rm lin}$,点線:$\Delta \hat{\omega}^{\rm lin}$ }
  \end{minipage}
  \begin{minipage}{0.32\linewidth}
    \centering
    \includegraphics[width = .99\linewidth]{figs/Esp}
    \subcaption{ 実線:$E^{\rm lin}$,点線:$\hat{E}^{\rm lin}$}
  \end{minipage}
  \caption{近似線形モデルと特異摂動近似システムの初期値応答(青:発電機1,赤:発電機2,黒:発電機3)}
  \label{fig:timeexsp}
  }
\end{figure}

式\ref{eq:pdsp}の条件に基づいて,特異摂動近似システムの同期がつぎのように特徴づけられる。

\begin{定理}[特異摂動近似システムの同期]\label{thm:2ndsys}
式\ref{eq:spamodel}の特異摂動近似システムを考える。ただし,入力は恒等的に0とする。
任意の初期値および任意の正定数$M_i$,$D_i$に対して,内部状態が式\ref{eq:Msp}の平衡点集合$\mathcal{M}_{\rm sp}$のいずれかの点に収束するための必要条件は,式\ref{eq:nescon}および
\begin{align}\label{eq:eigreal}
\bm{\Lambda}(L-CA^{-1}B)\subseteq [0,\infty)
\end{align}
が成り立つことである。
ただし,$\bm{\Lambda}(\cdot)$は固有値の集合を表す。
特に,送電網が無損失であるとき,上記の収束性の必要十分条件は,式\ref{eq:nescon}および式\ref{eq:eigreal}が成り立つことである。
\end{定理}

\begin{証明}
まず,送電網が無損失である場合に,式\ref{eq:nescon}と式\ref{eq:eigreal}が成り立つことが,題意の収束性の十分条件であることを示す。
式\ref{eq:spamodel}の特異摂動近似システムは,式\ref{eq:trFs}の$F(s)$と
\begin{align*}
G_{\rm sp}(s):= \frac{1}{s} (L-CA^{-1}B)
\end{align*}
のネガティブ・フィードバック系として解釈できる。
送電網が無損失であるとき,$L-CA^{-1}B$は対称であるため,式\ref{eq:eigreal}の条件は式\ref{eq:pdsp}の条件,すなわち,$G_{\rm sp}(s)$が正実であることと等価である。
したがって,式\ref{eq:pdsp}が成り立つならば,例\ref{ex:entsys}の議論と同様にして,そのネガティブ・フィードバック系が内部安定であることが示される。
この内部安定性は,式\ref{eq:spamodel}から冗長な積分器を削除した
\begin{align*}
\spliteq{
\mat{
\dot{\hat{\delta}}^{\rm lin}_e \\
\sfdiag(M_i) \Delta \dot{\hat{\omega}}^{\rm lin} 
}
&=
\mat{
 0 & \omega_0 W \\
  -(L_0 -CA^{-1}B_0) & -\sfdiag(D_i)  \\
 }
\mat{
\hat{\delta}^{\rm lin}_e \\
\Delta \hat{\omega}^{\rm lin}
}
\\
&+
\mat{
0 & 0 \\
I & CA^{-1}
}
\mat{
P_{{\rm mech}}^{\rm lin} \\
V_{{\rm field}}^{\rm lin}
}
}
\end{align*}
が漸近安定であることと等価である。
したがって,任意の初期値および任意の正定数$M_i$,$D_i$に対して,システムの状態は$\mathcal{M}_{\rm sp}$のいずれかの点に収束することが示される。

つぎに,式\ref{eq:nescon}が必要であることは補題\ref{lem:nescon}の議論と同様に明らかであるため,式\ref{eq:eigreal}が成り立つことが題意の収束性の必要条件であることを示す。
このために,$K:=L-CA^{-1}B$として,つぎの2つの場合に分けて議論する。
\begin{itemize}
\item[(a)] $K$の固有値のうち,実部が負であるもの,または,純虚数であるのものが存在する。
\item[(b)] $K$の固有値のうち,実部が正であり,かつ,虚部が非零であるものが存在する。
\end{itemize}
まず,(a)である場合に,ある正定数$M_i$,$D_i$が存在して,式\ref{eq:spamodel}の特異摂動近似システムが不安定となることを示す。
以下では,すべての$i$に対して$M_i=1$,$D_i=d$と選ぶ。
また,正の定数倍に関してシステムの安定性は不変であるから,$\omega_0 =1$としても一般性は失われない。
このとき,式\ref{eq:spamodel}の固有方程式は
\begin{align*}
\mat{
0 & I \\
-K & -d I
}
\mat{v\\w}
=
\lambda \mat{v\\w}
\end{align*}
である。
この方程式から$w$を代入により消去すれば
\begin{align*}
\left(\lambda^2 I +d \lambda I + K
\right) v =0
\end{align*}
が得られる。
この固有方程式は,$v$が$K$の固有ベクトルであり,その固有値$\kappa$に対して
\begin{align*}
\lambda^2 + d\lambda +\kappa =0
\qquad
\Longleftrightarrow
\qquad
\lambda = \frac{-d \pm \sqrt{d^2-4\kappa} }{2}
\end{align*}
が成り立つことを意味する。
したがって,(a)である場合に,$\sqrt{d^2 - 4\kappa }$の実部が$d$より大きいことを示せば良い。
一般に,任意の複素数$\bm{z}$に対して
\begin{align*}
\real[\bm{z}] = \sqrt{ \real[\bm{z}^2 ] + (\imag[\bm{z}])^2 }
\end{align*}
と表せることから,$\bm{z} = \sqrt{d^2 - 4\kappa }$とすると
\begin{align*}
\real \Bigl[
\sqrt{d^2 - 4\kappa }
\Bigr]
=\sqrt{
d^2 - 4 \real[\kappa]
+
(\imag[ \bm{z} ])^2
}
\end{align*}
が得られる。
この値は,(a)である場合,すなわち,$\kappa$の実部が負,または,$\kappa$の実部が0かつ$\kappa$の虚部が非零である場合には,必ず$d$より大きい。
したがって,式\ref{eq:spamodel}の特異摂動近似システムは不安定である。

つぎに,(b)である場合を考える。
いま,$K$は実行列であるため,それが複素数の固有値をもつ場合には,虚部が負のものが必ず存在する。
その固有値を$\kappa$と表す。
このとき,$G_{\rm sp}(\bm{j} \omega)$は$\sigma (\bm{j} \omega):= \frac{\kappa}{\bm{j} \omega}$を固有値にもつ。
いま,ある適当な$\hat{\omega}\in (0,\infty)$に対して
\begin{align}\label{eq:Fspara}
-\real[\sigma(\bm{j} \hat{\omega})]  = \frac{D}{\omega_0}
,\qquad
-\imag[\sigma(\bm{j} \hat{\omega})]  = \frac{\hat{\omega} M}{\omega_0}
\end{align}
となるように定数$D$,$M$を選び,すべての$i$に対して$M_i=M$,$D_i=D$とすれば
\begin{align*}
F(\bm{j} \hat{\omega}) = - \frac{1}{ \sigma(\bm{j} \hat{\omega}) } I
\end{align*}
と表すことができる。
なお,$\kappa$の実部が正であり,虚部が負であることから,$D$と$M$は正である。
このとき,固有値$\sigma (\bm{j} \omega)$に対する$G_{\rm sp}(\bm{j} \hat{\omega})$の固有ベクトルを$v(\bm{j} \hat{\omega})$とすれば
\begin{align*}
\bigl\{ I+G_{\rm sp}(\bm{j} \hat{\omega})F(\bm{j} \hat{\omega})\bigr\}
v(\bm{j} \hat{\omega})
=0
\end{align*}
が得られる。
これは$\sfdet [I+G_{\rm sp}(\bm{j} \hat{\omega}) F(\bm{j} \hat{\omega})]=0$であることを意味する。
したがって,$F(s)$と$G_{\rm sp}(s)$のネガティブ・フィードバック系は内部安定ではない。
\end{証明}




定理\ref{thm:2ndsys}は,送電網が無損失であるとき,内部電圧の時定数が十分に小さい極限では,式\ref{eq:nescon}および式\ref{eq:eigreal}が成り立つことが,任意の慣性定数と制動係数に対して式\ref{eq:lindyn}の線形近似システムが同期を達成するための必要十分条件であることを意味している。
また,それら2つの条件は,送電網が無損失であるかどうかに関わらず,同期のための必要条件であることも示している。
特に,式\ref{eq:eigreal}の必要条件は,$L-CA^{-1}B$の固有値が非負の実数でない場合には,線形近似システムの同期が達成されないような発電機の定数が必ず存在することを示している。
本節の結論はつぎの定理にまとめられる。


\begin{定理}[線形近似システムの同期条件]\label{thm:sync}
式\ref{eq:lindyn}の線形近似システムを考える。
任意の初期値および任意の正定数$M_i$,$D_i$,$\tau_i$に対して,線形近似システムが同期するための必要条件は,
\begin{itemize}
\item[(i)] $A$が安定であること,かつ
\item[(ii)] 式\ref{eq:nescon}が成り立つこと,かつ
\item[(iii)] 式\ref{eq:eigreal}が成り立つこと
\end{itemize}
である。
特に,送電網が無損失であるとき,上記の収束性の必要十分条件は,条件(i)--(iii)が成り立つことである。
\end{定理}

\begin{証明}
送電網が無損失である場合に,条件(i)--(iii)が成り立つならば線形近似システムの同期が達成されることは,これまでの議論で示されている。
一方で,条件(ii)の必要性は補題\ref{lem:nescon}の議論で示されている。
また,条件(i)と条件(iii)は,定理\ref{thm:2ndsys}に示されているように,すべての$\tau_{{\rm d}i}$が十分に小さい極限において,偏差サブシステムが漸近安定となるための必要条件である。
\end{証明}

定理\ref{thm:sync}は,送電網が無損失である場合に,任意の発電機の定数に対して式\ref{eq:lindyn}の線形近似システムの同期が達成されるための必要十分条件を与えている。
また,送電網が無損失であるかどうかに関わらず,条件(i)--(iii)がその必要条件であることも示している。
なお,条件(i)は発電機の同期リアクタンスの値に部分的に依存しているが,条件(ii)や条件(iii)は,送電網のアドミタンス行列,内部電圧と発電機偏角差の定常値のみに関するものであり,発電機の物理定数には依存しないことに注意されたい。
また,$L-CA^{-1}B$は,\ref{fig:blocklin}において,内部電圧の時定数$\tau_{{\rm d}i}$が十分に小さい場合の発電機偏角$\delta^{\rm lin}$と有効電力$P^{\rm lin}$の入出力関係を表す行列,すなわち
\begin{align*}
\hat{P}^{\rm lin}(t) = (L-CA^{-1}B) \hat{\delta}^{\rm lin}(t)
- C A^{-1} V^{\rm lin}_{\rm field}(t)
\end{align*}
の第1項目に現れる行列である。
ただし,特異摂動近似を適用した場合の信号であることを表すため$\hat{P}^{\rm lin}$として区別した。
この定理を用いて,以下のように近似線形モデルの同期を解析することができる。


\begin{figure}[t]
  \centering
  {
  \begin{minipage}{0.32\linewidth}
    \centering
    \includegraphics[width = .85\linewidth]{figs/gam01thm}
    \subcaption{ $\gamma=0.1$ }
  \end{minipage}
  \begin{minipage}{0.32\linewidth}
    \centering
    \includegraphics[width = .85\linewidth]{figs/gam2thm}
    \subcaption{ $\gamma=2$ }
  \end{minipage}
  \begin{minipage}{0.32\linewidth}
    \centering
    \includegraphics[width = .85\linewidth]{figs/gam5thm}
    \subcaption{ $\gamma=5$ }
  \end{minipage}
  \caption{例\ref{ex:linsyssim}の定数に対する近似線形モデルの同期解析}
  \label{fig:gamthm}
  }
\end{figure}

\begin{figure}[t]
  \centering
  {
  \begin{minipage}{0.32\linewidth}
    \centering
    \includegraphics[width = .85\linewidth]{figs/gam01ex}
    \subcaption{ $\gamma=0.1$ }
  \end{minipage}
  \begin{minipage}{0.32\linewidth}
    \centering
    \includegraphics[width = .85\linewidth]{figs/gam2ex}
    \subcaption{ $\gamma=2$ }
  \end{minipage}
  \begin{minipage}{0.32\linewidth}
    \centering
    \includegraphics[width = .85\linewidth]{figs/gam5ex}
    \subcaption{ $\gamma=5$ }
  \end{minipage}
  \caption{定数の設定を変更した場合の近似線形モデルの同期解析}
  \label{fig:gamex}
  }
\end{figure}


\begin{例}[正実性や特異摂動近似に基づく近似線形モデルの同期解析]\label{ex:linthm}
定理\ref{thm:sync}を用いて,例\ref{ex:linsyssim}で扱った3つの発電機から構成される近似線形モデルの同期を解析してみよう。
発電機の物理定数などの設定はすべて例\ref{ex:linsyssim}と同じであるとする。
\ref{fig:gamsta}(a)--(c)に,定理\ref{thm:sync}で示される必要条件が満たされないパラメータの領域を重ねてプロットしたものが\ref{fig:gamthm}(a)--(c)である。
ただし,条件(i)と(ii)はすべてのパラメータに対して満たされていたため,条件(iii)について,$L-CA^{-1}B$の固有値に実部が負であるものが含まれる場合を赤で示し,複素数の固有値が含まれる場合を紫で示している。
この結果で注目すべき点はつぎの2つである。
\begin{itemize}
\item 赤で示されているパラメータ領域は,同期が達成される青いパラメータ領域の境界の一部を正確に捉えている。
\item 紫で示されているパラメータ領域は,同期が達成される青いパラメータ領域と重なる部分が存在している。
\end{itemize}
まず,1点目に関して,条件(iii)の必要条件は「システムが不安定化するような発電機の物理定数の設定が少なくとも1つは存在すること」を表していることに注意されたい。
したがって,例\ref{ex:linsyssim}で設定されている特定の定数に対しては,同期が達成されるパラメータ領域を必ずしも正確に捉えられるとは限らない。
一方で,実際に青い領域の境界の一部を正確に捉えられているという事実は,$L-CA^{-1}B$に実部が負の固有値が含まれる場合には,多くの場合で近似線形モデルが不安定となることを示唆している。
つぎに,2点目に関して,青の領域と紫の領域が重なっていることは,例\ref{ex:linsyssim}の物理定数の設定では同期が達成されている一方で,システムが不安定化するような他の定数が必ず存在することを意味している。
すなわち,青と紫が重なる領域は,発電機の物理定数などの値に依存して同期する場合としない場合が混在するパラメータ領域であるといえる。
なお,定理\ref{thm:sync}では,送電網が無損失であるとき,すなわち,$\theta_2=0$である横軸上のパラメータに対しては,赤ではない値に$\theta_1$を設定する限り,それらの定数の値に依らず近似線形モデルの同期が必ず達成されることも示されている。
一方で,定理\ref{thm:sync}により,定数に依存しない同期の達成を数学的に保証できるのは,この横軸上のパラメータのみであることに注意されたい。
それ以外のパラメータ領域における同期を数学的に解析するためには,「送電網が無損失である」という仮定からある程度の逸脱を許すように理論を一般化する必要がある。

参考として,定数の設定の一部を変更し
\begin{align*}
E_1^{\star}=2
,\quad
E_2^{\star}=4
,\quad
E_3^{\star}=6
,\quad
D_1 = 2
,\quad
D_2 = 1.5
,\quad
D_3 = 1
\end{align*}
とした場合の結果を\ref{fig:gamex}に示そう。
この図において現れている「白い領域」は,上記の定数の設定ではシステムが不安定である一方で,システムが不安定化するような定数が必ず存在することまでは数学的に証明されていないパラメータ領域である。
より具体的には,$L-CA^{-1}B$が非負の実数固有値しかもたないにも関わらず,近似線形モデルが何らかの理由で不安定化したパラメータ領域である。
このような,白い領域のパラメータ設定に対しては,他の定数を設定した場合に同期が達成されるか否かについて,以上の解析だけで結論を与えることはできない。
\end{例}

送電網が無損失でない場合に近似線形モデルの同期を特徴づける必要十分条件を導出することは難しい。
一方で,送電網が無損失であることが,式\ref{eq:trGs}の$G(s)$の正実性を特徴づける条件の1つであるという事実に基づいて,$G(s)$の正実な範囲からの逸脱を定量的に見積もることができれば,システムの同期が達成されるために発電機の物理定数などが満たすべき十分条件を求めることは可能である。





%\bibliographystyle{myjunsrt}		% bib style
%\bibliography{reference}	% your bib database


\newpage
\end{document}