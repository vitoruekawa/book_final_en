\documentclass[graybox, envcountchap]{svmult}
% Springer document settings
\usepackage[bottom]{footmisc}% places footnotes at page bottom

\usepackage{newtxtext}       % 
\usepackage[varvw]{newtxmath}       % selects Times Roman as basic font
%%%%%%%%%%%%%%%%%%%%%%%%%%%%%%%

% \usepackage{amssymb}
\usepackage{amsmath}
\usepackage{enumitem}


\usepackage{graphicx}
\usepackage{color}
\usepackage{cite}
\usepackage{makeidx}
\usepackage{ntheorem}

\usepackage{ascmac}
\usepackage{eclbkbox}
\usepackage{dsfont}

\usepackage{longtable}

\usepackage{url}

\usepackage{hyperref}

\usepackage{multicol}

%% --川口追加--
\makeatletter
\let\MYcaption\@makecaption
\makeatother
\usepackage{subcaption}
\captionsetup{compatibility=false}      % 必要に応じて

\makeatletter
\let\@makecaption\MYcaption
\makeatother
% ----

%%
\theoremstyle{plain}
\theoremheaderfont{\bfseries}
\theorembodyfont{\rmfamily}
\theoremseparator{\hspace{1ex}}
\theoremindent0cm
\theoremnumbering{arabic}
\theoremprework{\vspace{1ex}\begin{shadebox}\vspace{1ex}}
\theorempostwork{\vspace{-1ex}\end{shadebox}\vspace{1ex}}

%%
\theoremclass{theorem}

%%
\theoremclass{theorem}

%%
\theoremclass{theorem}


%%
\theoremstyle{break}
\theoremheaderfont{\bfseries}
\theorembodyfont{\rmfamily}
\theoremseparator{}
\theoremindent0cm
\theoremnumbering{arabic}
\theoremprework{\vspace{1.5ex}\begin{breakbox}\vspace{-0.5ex}}
\theorempostwork{\vspace{-0.5ex}\end{breakbox}\vspace{1.5ex}}

%%
\theoremstyle{nonumberplain}
\theoremseparator{\hspace{1ex}}

%%
\newtheorem{assumption}{Assumption}[section]

%%
\renewcommand{\theproblem}{}

\renewcommand{\theremark}{}


\newcommand{\red}[1]{{\color{red}#1}}
\newcommand{\blue}[1]{{\color{blue}#1}}
\newcommand{\green}[1]{{\color{green}#1}}

\DeclareMathOperator*{\argmax}{arg\,max}

\newcommand{\bm}[1]{\boldsymbol{#1}}
\newcommand{\sfT}{\mathsf{T}}

\newcommand{\advanced}{$^{\ddag}$}

\DeclareMathOperator{\sfsin}{\mathsf{sin}}
\DeclareMathOperator{\sfcos}{\mathsf{cos}}
\DeclareMathOperator{\sftan}{\mathsf{tan}}
\DeclareMathOperator{\sfarctan}{\mathsf{arctan}}

\DeclareMathOperator{\sfdiag}{\mathsf{diag}}
\DeclareMathOperator{\sfcol}{\mathsf{col}}
\DeclareMathOperator{\sfdet}{\mathsf{det}}
\DeclareMathOperator{\sfadj}{\mathsf{adj}}
\DeclareMathOperator{\sftrace}{\mathsf{trace}}

\DeclareMathOperator{\real}{\mathsf{Re}}

\DeclareMathOperator{\sfker}{\mathsf{ker}}
\DeclareMathOperator{\sfim}{\mathsf{im}}

\DeclareMathOperator{\sfdim}{\mathsf{dim}}
\DeclareMathOperator{\sfspan}{\mathsf{span}}

\DeclareMathOperator{\sfint}{\mathsf{int}}

\DeclareMathOperator*{\sfmin}{\mathsf{min}}
\DeclareMathOperator*{\sfmax}{\mathsf{max}}
\DeclareMathOperator*{\sfsup}{\mathsf{sup}}

\DeclareMathOperator{\sfsat}{\mathsf{sat}}

\newcommand{\mat}[1]{\left[\: \begin{matrix} #1 \end{matrix} \:\right]}
\newcommand{\spliteq}[1]{\begin{split} #1 \end{split}}
\newcommand{\simode}[1]{\begin{cases}  \begin{split} #1 \end{split} \end{cases}}

\newcommand{\proofend}{\hfill \rule{2mm}{3mm}}

\newcommand{\Xti}{X_i'}
\newcommand{\Xsi}{X_i}

\newcommand{\Xtone}{X_1'}
\newcommand{\XtN}{X_N'}

\newcommand{\Xt}{X'}
\newcommand{\Xs}{X}

\newcommand{\taudi}{\tau_i}
\newcommand{\taud}{\tau}

\newcommand{\Cgi}{b_i}


\newcommand{\Ifd}{I_{\rm field} }

\newcommand{\matlab}{\textsc{Matlab} }





%% --川口追加--
\newcommand{\thshift}{\theta_{12}}
\newcommand{\thshiftb}{\theta_{32}}
\newcommand{\Ysa}{\bm y_{12}}
\newcommand{\bca}{c_{12}}
\newcommand{\Ysb}{\bm y_{32}}
\newcommand{\bcb}{c_{32}}
\newcommand{\bcij}{c_{ij}}
\newcommand{\Is}{{\bm I}_{12}' }
\newcommand{\im}{\bm j}
\newcommand{\tr}{{\sf T}}

\begin{document}

\chapter{Introduction}

In this Chapter, the objectives, characteristics and instructions of how to use
this book will be explained. The mathematical notation used in this book will
also be summarized.


\section{Control of electrical power system}

\subsection{Importance of interdisciplinary integration aimed at power system
reform}

In Japan, the reformation of the power system began with the Great East Japan
Earthquake in March 2011, and the full-scale introduction of renewable energy,
such as solar, wind and geothermal power took place after the full
liberalization of the electricity retail market in April 2016. Furthermore, in
October 2020, Japan pledged to achieve carbon neutrality by 2050 as a matter of
national policy. Thus, improvements to the existing energy supply system and
maximization of the use of renewable energy are strongly required. To achieve
such an ambitious goal, technologies related to energy generation, as well as
digital technologies, such as telecommunications, must be fully utilized to
create a smart grid that improves the operation of power systems and to control
supply and demand of renewable energy, where power supply fluctuates with
weather.

A power system is a large-scale system that consists of the interaction of
events and phenomena on a wide range of spatio-temporal scales. Specifically,
the operation of actual power systems is influenced not only by the properties
of electromechanical devices and facilities, such as generators and transmission
lines, but also by natural phenomena, such as weather condition and natural
disasters, and human-related activities, such as electricity demand by the
consumers, and commercial activities of power generation, transmission and
distribution companies.

Therefore, the reformation of current power systems should be supported by the
expertise of various academic fields in addition to energy-related fields. For
instance, notions of economics are essential for designing a fair market system
capable of adding value to the planned and observed values of energy supply and
demand, and at the same time avoiding monopoly formation. In addition, to manage
the uncertainties of renewable energy generation, it is imperative to develop
methods to plan supply and demand and to forecast power generation from these
resources. For this purpose, notions of mathematical optimization and statistics
are fundamental. Finally, expertise in feedback control methods is essential for
improving system stability against disturbances such as load fluctuations and
ground faults. 

In summary, the design and operation of power systems require expertise of
various research fields. This academic diversity will lead to challenging yet
attractive research and development topics.


\subsection{Objective of this book}

Control systems engineering is an interdisciplinary field that covers many areas
of science, such as control theory, information theory, data science, systems
engineering and mathematical optimization, and it is expected to contribute to
the research and development of power systems. However, since power systems are
complex systems composed by a wide range of devices and phenomena, it is often
very challenging for beginners to get an overview of the whole system. This
difficulty hinders the contribution of experts in other areas in research and
development of power systems. This book attempts to tackle this problem by
providing an introduction and an overview of power systems.

By explaining modeling, numerical simulation, control system design and
mathematical analysis for power systems will be explained from a viewpoint of
network system analysis and control, this book aims to make students and
researchers in the field of system control to be able to:

\begin{itemize}
	\item understand the structure of power systems and mathematical foundation
	in the language of control systems engineering, and
	\item build a numerical simulation environment for analysis and control of
	power systems on their own. 
\end{itemize}

For this purpose, the structure and characteristics of power systems will be
explained by using fundamental concepts of control systems engineering, such as
state-space representation, equilibrium point, stability and feedback control.
In addition, the numerical simulation enviornment will be conducted based on the
foundations of object-oriented programming, describing the entire complex power
system as classes and methods and providing code examples for easier
understanding.

\subsection{Characteristics of this book}
\red{Review English in this section}
There are many good books, both in Japanese and other languages, on electrical
power systems. Characteristics of this book compared to these books are as
follows:

\begin{itemize}
	\item one can follow this book without a background in electromagnetism, and
	\item the structure and characteristics of power systems as a network system
	are clearly described.
\end{itemize}

For example, a standard book on electrical power system engineering assumes that
one has knowledge on three-phase electric power and electromagnetic induction,
and focuses on the explanation of electromagnetic phenomena inside of each
generator and so on. Meanwhile, as far as the authors are aware, there are few
if any books that clearly explain the dynamic characteristics and structures of
the whole system when multiple generators are connected via a power grid from
the viewpoint of control systems engineering.  In addition, while a model in
which multiple generators are connected in a power grid is used to analyze power
transmission distribution, known as power flow calculation, in system stability
discussions, different models are often introduced depending on the topic of
discussions (e.g., single machine infinite bus system models, models consisting
of only one generator).  This makes it difficult for novice learners who are
unfamiliar with electrical power system engineering to see the link between
explained topics and gain a comprehensive view of the entire power system.
Considering these situations, this book presents essential basics for students,
researchers, and technologists in the field of system control to “understand”
the structure and characteristics of power systems “based on system theory.”

As discussed above, various academic knowledge is required to design and operate
power systems.  With the help of the information in this book, we hope that
challenging and attractive research and development targets will be used as one
of the benchmark models in the field of system control.

\section{How to use this book}

\subsection{Overall structure}
\red{Review English in this section}
This book introduces mathematical models of electrical power systems in Chapter
2, then explains the steps for numerical simulation of the electrical power
system model in Chapter 3.  This is followed by discussions of the stability of
the electrical power system model in Chapter 4 and control methods to improve
the system stability in Chapter 5.  Finally, in Chapter 6, we present the
results of numerical simulations using a large-scale electrical power system
model.

This book assumes that readers will follow from Chapter 2 to Chapter 6 in that
order.  However, readers who aim to perform numerical simulations on their own
can skip Sections with “\advanced” in the title.  These Sections primarily
describe development topics that are useful in understanding the mathematical
structure of the electrical power system model.  Since this book focuses on
explanations of the basics of electrical power system analysis, topics related
to new energy, such as solar power, wind power, and batteries are not presented.
Please refer to \cite{sadamoto2019dynamic} for these topics.

\subsection{Published numerical simulation codes and supplementary material}
\red{Review English in this section}
\red{ The numerical simulations described in this document were created with a
Matlab program called GUILDA (Grid \& Utility Infrastructure Linkage Dynamics
Analyzer), which is being developed mainly by the authors.
%(ЈFIGref{fig:guilda}). GUILDA is available on GitHub as open source software
and can be freely extended by users. The programs used for the numerical
simulations in this book are also available. For details, please refer to the
web page of this book by Corona Inc. for more details. Color versions of the
figures in this book are also provided.  Translated with
www.DeepL.com/Translator (free version) }


%\begin{figure}[t]
%\centering
%\includegraphics[width = .30\linewidth]{figs/guilda}
%\caption{\textbf{GUILDAロゴ}}
%\label{fig:guilda}
%\medskip
%\end{figure}

\subsection{Mathematical notations}

Real numbers and complex numbers are expressed as $\mathbb{R}$ and $\mathbb{C}$,
respectively. Real vectors with $n$ dimensions are expressed as $\mathbb{R}^n$,
while real matrices with dimensions $n\times m$  are expressed as
$\mathbb{R}^{n\times m}$. The same style of notation is used for complex vectors
and matrices. In addition, the imaginary unit is expressed as $\bm{j}$, and bold
font is used for complex scalars, vectors and matrices.

A vector in which all elements are one is expressed as $\mathds{1}$. A diagonal
matrix that has real scalar, $c_1,\ldots,c_n$, as diagonal elements is expressed
as $(c_i)_{i\in\{1,\ldots, n\}}$. When it is clear from the context, subscripts
are omitted and the diagonal matrix is simply expressed as $\sfdiag(c_i)$. The
same applies to complex numbers. In addition, the inverse matrix of a
nonsingular complex matrix $\bm{Z}$ is expressed as $\bm{Z}^{-1}$.

The transposed matrix of real matrix $A$ is expressed as $A^{\sf T}$. The real
and imaginary parts of a complex matrix $\bm{Z}$ are expressed as
$\real[\bm{Z}]$ and $\imag[\bm{Z}]$, respectively. In other words, for an
arbitrary complex matrix $\bm{Z}$:

\[
\bm{Z} = \real \left[\bm{Z} \right] + \bm{j} \imag \left[\bm{Z} \right]
\]

In addition, the conjugate of a complex matrix $\bm{Z}$ is expressed as
$\overline{\bm{Z}}$, and the conjugate of its transpose is expressed as
$\bm{Z}^*$. In other words:

\[
\overline{\bm{Z}} = \real \left[\bm{Z} \right] - \bm{j} \imag \left[\bm{Z} \right]
,\qquad
\bm{Z}^* = \left(\real \left[\bm{Z} \right] \right)^{\sf T} - \bm{j} 
\left( \imag \left[\bm{Z} \right] \right)^{\sf T}
\]

The absolute value of a complex scalar $\bm{z}$ is expressed as $|\bm{z}|$,
while its argument is expressed as $\angle \bm{z}$.  In other words, $\bm{z} =
|\bm{z}|e^{\bm{j} \angle \bm{z}}$.

When complex symmetric matrix $\bm{Z}=\bm{Z}^*$ is positive definite, it is
written as $\bm{Z}\succ 0$. When $\bm{Z}$ is positive semi-definite, it is
written as $\bm{Z}\succeq 0$. Similarly, negative definite and negative
semi-definite are expressed with inequality signs of the opposite direction.

% \footnote{
% When $\bm{x}^{*} \bm{A} \bm{x} \geq 0$ is true for arbitrary complex vector $\bm{x}$, 
% square matrix $\bm{A}$ is called \textbf{ positive semi-definite}. 
% When x*Ax > 0 is true for arbitrary vector $\bm{x}\neq 0$, $\bm{A}$ is called \textbf{positive definite}.
% This is the same for negative semi-definite and negative definite if the sign is inverted. 
% The condition necessary for a complex symmetric matrix $\bm{A}$ to be positive semi-definite is that all eigenvalues be non-negative. 
% Similarly, the condition necessary for $\bm{A}$ to be positive definite is for all eigenvalues to be positive.
% }

\begin{COLUMN}
\noindent \textbf{Singularity of matrix}:
A square matrix $\bm{A}$ is said to be \textbf{regular}(nonsingular) if it has
an inverse $\bm{A}^{-1}$ such that:
\[
\bm{A}\bm{A}^{-1}=I
\]
The above condition is true if and only if the determinant of $\bm{A}$ is
non-zero.
\smallskip

\noindent \textbf{Positive and negative definiteness of matrices}:
For a given complex vector $\bm{x}$, a square matrix $\bm{A}$ is said to be
\textbf{positive semi-definite} if the following relationship is true:
\[
\bm{x}^{*} \bm{A} \bm{x} \geq 0
\]

Additionally, $\bm{A}$ is \textbf{positive definite} if $\bm{x}\neq 0$ and: 
\[
\bm{x}^* \bm{A}\bm{x} > 0
\]

The definition of semi-negative and negative definiteness are similar, with the
only difference being the sign of the inequality. A necessary and sufficient
condition for a complex symmetric matrix $\bm{A}$ to be positive semi-definite
is that all its eigenvalues are non-negative. Similarly, a necessary and
sufficient condition for $\bm{A}$ to be positive definite is that all its
eigenvalues are strictly positive.

\end{COLUMN}

The null space of real matrix $A\in \mathbb{R}^{n\times m}$, $\sfker A$, is
defined as:

\[
\sfker A:= \left\{
x\in \mathbb{R}^{m} : Ax =0
\right\}
\]

where the symbol “$:=$” means that the term on the right side of the symbol
defines the term on its left side. The null space of a complex matrix $\bm{Z}
\in \mathbb{C}^{n\times m}$ is similarly defined as:

\[
\sfker \bm{Z}:= \left\{
\bm{x}\in \mathbb{C}^{m} : \bm{Z}\bm{x} =0
\right\}
\]

Moreover, given a set of real vectors $v_1,\ldots,v_m\in \mathbb{R}^{n}$, the
linear space formed by all the vectors that can be written as linear combination
of vectors $v_1,\ldots,v_m$ is defined as:
\[
\sfspan\{v_1,\ldots,v_m\}:=\left\{
c_1 v_1+\cdots+c_m v_m : 
(c_1,\ldots,c_m)\in \mathbb{R}\times \cdots \times \mathbb{R}
\right\}
\]

Finally, the Euclidean norm of a real vector $x\in \mathbb{R}^n$ is represented
by $\|x\|$. In other words:
\[
\|x\| := \sqrt{|x_1|^2+\cdots+|x_n|^2}
\]
where the $x_i$ is thee $i$th element of $x$.

Please note that, in this book, standard symbols in control systems and
electrical power system engineering might overlap. For example, in control
systems engineering, “$G$” is often used to represent a system, whereas in
electrical power system engineering, it is commonly used to express conductance.
Unless there is a concern about misunderstanding of the context, overlaps
between these symbols will not be discussed.

\newpage
\end{document}